\pdfminorversion=4
\documentclass[9pt]{beamer} 
\setbeamertemplate{navigation symbols}{}
%\setbeamertemplate{frametitle}{}

%\usetheme{Montpellier}
\setbeamertemplate{footline}[frame number]

\newcounter{multipleslide}
\makeatletter%
\newcommand{\multipleframe}{%
\setcounter{multipleslide}{\value{framenumber}}
\stepcounter{multipleslide}
\patchcmd{\beamer@@tmpl@footline}% <cmd>
  {\insertframenumber}% <search>
  {\themultipleslide}% <replace>
  {}% <success>
  {}% <failure>
}
\newcommand{\restoreframe}{%
\patchcmd{\beamer@@tmpl@footline}% <cmd>
  {\themultipleslide}% <search>
  {\insertframenumber}% <replace>
  {}% <success>
  {}% <failure>
\setcounter{framenumber}{\value{multipleslide}}%
}
\makeatother%

\usepackage{pgfpages}
\usepackage{booktabs}
\usepackage[%				
%   group-separator = {.}, 			 % group-seperator={,} -> 1,345,234.23
%   round-mode 		= places,		 % round-mode= places(2digits), figures(1digit)
%   round-precision = 3,			 % round-precision= x , xdigits are rounded to
	binary-units 	= true,			 % Laden von \byte \bit, \kibi usw. (false default)
	locale = DE,					 % lacale= DE, uses german
	per=slash,
%	loctolang={UK:english, DE:ngerman},	% loctolang={USA:USenglish,DE:ngerman}, if language changed in
									 % document
%			detect-all,				 % detect-all : uses the text options not the math mode to set
									 % the letters/numbers
			]{siunitx}				 % For typical Units in Text- and Mathmode
%%%%%%%%%%%%%%%%%%%%%%%%%%%%%%%%%%%%%%%
% Loading mathematical writing style  %
%%%%%%%%%%%%%%%%%%%%%%%%%%%%%%%%%%%%%%%
\usepackage{mathtools}
    
\mode<presentation>
{
%\usetheme{Madrid}
\usetheme{Boadilla}

\setbeamercovered{transparent}
}
\usepackage{subfigure} 
\usepackage{amsmath}  			% erleichtert Mathe 
\usepackage{enumerate}			% schicke Nummerierung
\usepackage{graphicx} 			% für Grafik-Einbindung
\usepackage{multicol}
\usepackage{lmodern}
\usepackage[ngerman]{babel}
\usepackage[utf8]{inputenc}

\usepackage{media9}
%\usepackage{multimedia}

%=================== Set Beamer Templates ====================================
%\definecolor{DOGGYbg}{RGB}{167,183,208} %HAW Color 
%\definecolor{DOGGY}{RGB}{26,60,116} % HAW Color 2
%\definecolor{DOGGYbg}{RGB}{240,240,255} %HAW Color 
%\definecolor{DOGGY}{RGB}{26,60,116} % HAW Color 2
%\definecolor{dgreen}{RGB}{0,150,0} % HAW Color 2
%\definecolor{lightgray}{RGB}{150,150,150} % HAW Color 2

%\setbeamercolor*{structure}{fg=DOGGY, bg=DOGGYbg}
%\setbeamercolor*{palette primary}{use=structure,fg=structure.bg,bg=structure.fg!95!black}
%\setbeamercolor*{palette secondary}{use=structure,fg=structure.bg,bg=structure.fg!70!black}
%\setbeamercolor*{palette tertiary}{use=structure,fg=structure.bg,bg=structure.fg!95!black}
%\setbeamercolor*{palette quaternary}{fg=structure.bg,bg=black}
%\setbeamercolor*{block title}{use=structure,fg=structure.bg,bg=structure.fg!75!black}

\definecolor{DOGGYbg}{RGB}{80,90,100} %HAW Color
\definecolor{DOGGY}{RGB}{26,60,116} % HAW Color 2
\definecolor{lgray}{RGB}{240,240,240} %HAW Color
\definecolor{darkgreen}{RGB}{50,150,50} %HAW Color

\setbeamercolor*{structure}{fg=DOGGY, bg=DOGGYbg} 
\setbeamercolor*{palette primary}{use=structure,fg=structure.fg,bg=lgray}
\setbeamercolor*{palette secondary}{use=structure,fg=structure.fg,bg=lgray!90!black}
\setbeamercolor*{palette tertiary}{use=structure,fg=structure.fg,bg=lgray}
\setbeamercolor*{palette quaternary}{fg=structure.fg,bg=lgray}
\setbeamercolor*{block title}{use=structure,fg=structure.fg,bg=lgray}


\setbeamertemplate{navigation symbols}{} % No navigation symbols
\setbeamercolor{navigation symbols}{use=structure, fg=structure.bg,
bg=structure.bg}
% New footline
\defbeamertemplate{footline}{rownav}{%
  \leavevmode%
  \hbox{%
  \begin{beamercolorbox}[wd=.15\paperwidth,ht=2.5ex,dp=1ex,center]{author in
  head/foot}% only put the author and dont use the institute name - width 1/4
    \usebeamerfont{author in head/foot}\insertshortauthor
  \end{beamercolorbox}%
  \begin{beamercolorbox}[wd=.60\paperwidth,ht=2.5ex,dp=1ex,center]{title in
  head/foot}%
    \usebeamerfont{title in head/foot}\insertshorttitle
  \end{beamercolorbox}%
  \begin{beamercolorbox}[wd=.08\paperwidth,ht=2.5ex,dp=1ex,right]{date in
  head/foot}%
    \usebeamercolor[fg]{navigation symbols}% 
    \insertslidenavigationsymbol\hspace*{1ex}%
    \vspace*{-0.5ex}
  \end{beamercolorbox}%
  \begin{beamercolorbox}[wd=.17\paperwidth,ht=2.5ex,dp=1ex,center]{date in
  head/foot}%
    \usebeamerfont{date in head/foot}\insertshortdate{}\hspace*{2em}
    \textbf{\insertframenumber{}}% 
  \end{beamercolorbox}%
    }%
  \vskip0pt%
}
\setbeamertemplate{footline}[rownav]{}

\defbeamertemplate*{title page}{myPage}[1][]
{
  \vbox{}
  \vfill
  \begin{centering}
    \begin{beamercolorbox}[sep=8pt,center,#1]{author}
      \usebeamerfont{title}\inserttitle
    \end{beamercolorbox}%
    \vskip1em\par
    \begin{beamercolorbox}[sep=8pt,center,#1]{author}
      \usebeamerfont{author}\insertauthor
    \end{beamercolorbox}
    \begin{beamercolorbox}[sep=8pt,center,#1]{institute}
      \usebeamerfont{institute}\insertinstitute
    \end{beamercolorbox}
    \begin{beamercolorbox}[sep=8pt,center,#1]{date}
      \usebeamerfont{date}\insertdate
    \end{beamercolorbox}
  \end{centering}
  \vfill
}
 
\setbeamertemplate{title page}[myPage]{}

\newcommand{\boxHeightL}{7}
\newcommand{\boxWidthL}{7}
\newcommand{\boxHeightS}{7}
\newcommand{\boxWidthS}{7}
\newcommand{\boxWidthWide}{7}
\newcommand{\boxHeightHigh}{7}



\newcommand{\myboxOnePos}{\colorbox{\boxColorForOne}{\makebox(\boxWidthL,\boxHeightL){\textcolor{\textColorForOne}{}}}}
\newcommand{\myboxOneNeg}{\colorbox{\boxColorForOne }{\makebox(\boxWidthL,\boxHeightL){\textcolor{\textColorForOne}{-}}}}
\newcommand{\myboxZero}  {\colorbox{\boxColorForZero}{\makebox(\boxWidthL,\boxHeightL){\textcolor{\textColorForZero}{}}}}
\newcommand{\myboxSqrtPos}{\colorbox{\boxColorForSqrt}{\makebox(\boxWidthL,\boxHeightL){\textcolor{\textColorForSqrt}{}}}}
\newcommand{\myboxSqrtNeg}{\colorbox{\boxColorForSqrt}{\makebox(\boxWidthL,\boxHeightL){\textcolor{\textColorForSqrt}{-}}}}

\newcommand{\myBlackBox}{\colorbox{black}{\makebox(\boxWidthS,\boxHeightS){\textcolor{\textColorForSqrt}{}}}}
\newcommand{\myGrayBox}{\colorbox{gray}{\makebox(\boxWidthS,\boxHeightS){\textcolor{\textColorForSqrt}{}}}}
\newcommand{\myLightgrayBox}{\colorbox{lightgray}{\makebox(\boxWidthS,\boxHeightS){\textcolor{\textColorForSqrt}{}}}}
\newcommand{\myDarkgrayBox}{\colorbox{darkgray}{\makebox(\boxWidthS,\boxHeightS){\textcolor{\textColorForSqrt}{}}}}

\newcommand{\myBlackBoxHigh}{\colorbox{black}{\makebox(\boxWidthS,\boxHeightHigh){\textcolor{\textColorForSqrt}{}}}}
\newcommand{\myBlackBoxWide}{\colorbox{black}{\makebox(\boxWidthWide,\boxHeightS){\textcolor{\textColorForSqrt}{}}}}
\newcommand{\myLightgrayBoxHigh}{\colorbox{lightgray}{\makebox(\boxWidthS,\boxHeightHigh){\textcolor{\textColorForSqrt}{}}}}
\newcommand{\myLightgrayBoxWide}{\colorbox{lightgray}{\makebox(\boxWidthWide,\boxHeightS){\textcolor{\textColorForSqrt}{}}}}

\usepackage{tcolorbox} 

\newtcolorbox{mytextbox}[1]{%
    tikznode boxed title,
    enhanced,
    arc=0mm,
    interior style={white},
    attach boxed title to top left= {xshift=0.5cm, yshift=-\tcboxedtitleheight/2},
    fonttitle=\bfseries,
    colbacktitle=white,coltitle=black,
    boxed title style={size=normal,colframe=gray,boxrule=0pt},
    title={#1}}
\usepackage{color}				% to use colors, espacially for source codes
\usepackage[%
%	    table				% Zum automatischen Laden des Pakets colortbl - fuer Tabellen
	   ]{xcolor}				% for Hyperref package, that one can say red than rgb values

	   
	   
\definecolor{mygreen}{rgb}{0,0.6,0}
\definecolor{forestgreen}{rgb}{0.0, 0.27, 0.13}
\definecolor{mygray}{rgb}{0.5,0.5,0.5}
\definecolor{mymauve}{rgb}{0.58,0,0.82}
\definecolor{lila}{rgb}{0.6, 0.4, 0.8}
\definecolor{lavendel}{rgb}{0.75, 0.58, 0.89}
\definecolor{applegreen}{rgb}{0.55, 0.71, 0.0}
\definecolor{azure}{rgb}{0.0, 0.5, 1.0}
\definecolor{yellow}{rgb}{0.99, 0.93, 0.0}
\definecolor{brown}{rgb}{0.59, 0.29, 0.0}
\definecolor{cinnamon}{rgb}{0.82, 0.41, 0.12}
\definecolor{pumpkin}{rgb}{1.0, 0.46, 0.09}
\definecolor{pink}{rgb}{0.91, 0.25, 0.78}

\definecolor{DOGGYbg}{RGB}{80,90,100} %HAW Color
\definecolor{DOGGY}{RGB}{26,60,116} % HAW Color 2
\definecolor{lgray}{RGB}{240,240,240} %HAW Color
\definecolor{darkgreen}{RGB}{50,150,50} %HAW Color

\definecolor{brightgray}{RGB}{230,230,230}

\newcommand{\boxColorForOne}{black}
\newcommand{\boxColorForZero}{lightgray}
\newcommand{\boxColorForSqrt}{green}

\newcommand{\textColorForOne}{white}
\newcommand{\textColorForZero}{black}
\newcommand{\textColorForSqrt}{black}

\usepackage{tikz}
%\usepackage{background}
\usetikzlibrary{thedecorations, arrows,shapes, automata, positioning, calc, backgrounds, decorations.pathreplacing}


\newcommand{\tikzmark}[1]{\tikz[overlay, remember picture] \coordinate (#1);}

\tikzset{node/.style={draw,shape=circle}}
\newcommand{\order}[2][th]{\ensuremath{{#2}^{\mathrm{#1}}}}
\usepackage{graphicx} 				% für Grafik-Einbindung
%\usepackage{subfigure}				% Für Untergrafiken, wenn eine eigene Bildunterschrift erwuenscht
\usepackage{subcaption}
\usepackage[hypcap=false]{caption}				% für Captions bei Grafiken ohne figure Umgebung wie in Minipage nötig
\usepackage[%
	    update				% if pdf conversion is older than eps file, new conversion is done
	   ]{epstopdf}				% eps to pdf needs for pdflatex - after graphicx

%\usepackage{subfig}
\usepackage{capt-of}

%---------------------------------------------------------------------------------------------------
% Abbildungsverzeichnis
%---------------------------------------------------------------------------------------------------
%\graphicspath{{img/}}

\usepackage{listings}					% options : basicstyle= e.g. \ttfamily
							% fontadjust=true % before load package [scaled=0.78]{luximono}
							% columns=flexible

\lstset{ %
  backgroundcolor=\color{white},   % choose the background color; you must add \usepackage{color} or \usepackage{xcolor}; should come as last argument
  basicstyle=\footnotesize,        % the size of the fonts that are used for the code
  breakatwhitespace=false,         % sets if automatic breaks should only happen at whitespace
  breaklines=true,                 % sets automatic line breaking
  captionpos=b,                    % sets the caption-position to bottom
  commentstyle=\color{mygreen},    % comment style
  deletekeywords={...},            % if you want to delete keywords from the given language
  escapeinside={\%*}{*)},          % if you want to add LaTeX within your code
  extendedchars=true,              % lets you use non-ASCII characters; for 8-bits encodings only, does not work with UTF-8
  frame=single,	                   % adds a frame around the code
  keepspaces=true,                 % keeps spaces in text, useful for keeping indentation of code (possibly needs columns=flexible)
  keywordstyle=\color{blue},       % keyword style
  language=VHDL,                 % the language of the code
  morekeywords={*,...},            % if you want to add more keywords to the set
  numbers=none,                    % where to put the line-numbers; possible values are (none, left, right)
  numbersep=5pt,                   % how far the line-numbers are from the code
  numberstyle=\tiny\color{mygray}, % the style that is used for the line-numbers
  rulecolor=\color{black},         % if not set, the frame-color may be changed on line-breaks within not-black text (e.g. comments (green here))
  showspaces=false,                % show spaces everywhere adding particular underscores; it overrides 'showstringspaces'
  showstringspaces=false,          % underline spaces within strings only
  showtabs=false,                  % show tabs within strings adding particular underscores
  stepnumber=2,                    % the step between two line-numbers. If it's 1, each line will be numbered
  stringstyle=\color{mymauve},     % string literal style
  tabsize=2,	                   % sets default tabsize to 2 spaces
  title=\lstname                   % show the filename of files included with \lstinputlisting; also try caption instead of title
  literate=
  {á}{{\'a}}1 {é}{{\'e}}1 {í}{{\'i}}1 {ó}{{\'o}}1 {ú}{{\'u}}1
  {Á}{{\'A}}1 {É}{{\'E}}1 {Í}{{\'I}}1 {Ó}{{\'O}}1 {Ú}{{\'U}}1
  {à}{{\`a}}1 {è}{{\`e}}1 {ì}{{\`i}}1 {ò}{{\`o}}1 {ù}{{\`u}}1
  {À}{{\`A}}1 {È}{{\'E}}1 {Ì}{{\`I}}1 {Ò}{{\`O}}1 {Ù}{{\`U}}1
  {ä}{{\"a}}1 {ë}{{\"e}}1 {ï}{{\"i}}1 {ö}{{\"o}}1 {ü}{{\"u}}1
  {Ä}{{\"A}}1 {Ë}{{\"E}}1 {Ï}{{\"I}}1 {Ö}{{\"O}}1 {Ü}{{\"U}}1
  {â}{{\^a}}1 {ê}{{\^e}}1 {î}{{\^i}}1 {ô}{{\^o}}1 {û}{{\^u}}1
  {Â}{{\^A}}1 {Ê}{{\^E}}1 {Î}{{\^I}}1 {Ô}{{\^O}}1 {Û}{{\^U}}1
  {œ}{{\oe}}1 {Œ}{{\OE}}1 {æ}{{\ae}}1 {Æ}{{\AE}}1 {ß}{{\ss}}1
  {ű}{{\H{u}}}1 {Ű}{{\H{U}}}1 {ő}{{\H{o}}}1 {Ő}{{\H{O}}}1
  {ç}{{\c c}}1 {Ç}{{\c C}}1 {ø}{{\o}}1 {å}{{\r a}}1 {Å}{{\r A}}1
  {€}{{\euro}}1 {£}{{\pounds}}1 {«}{{\guillemotleft}}1
  {»}{{\guillemotright}}1 {ñ}{{\~n}}1 {Ñ}{{\~N}}1 {¿}{{?`}}1
}
 
\usepackage[
	    official,
	    right				% Position of the Euro-Sign. verwendung: \euro oder \EUR{123,45}
	   ]{eurosym}


% Loading unit options                

\usepackage[%				
%	    group-separator 	= {.}, 		% group-seperator={,} -> 1,345,234.23
%   	    round-mode 		= places,		% round-mode= places(2digits), figures(1digit)
%	    round-precision 	= 3,		% round-precision= x , xdigits are rounded to
	    binary-units 	= true,		% Laden von \byte \bit, \kibi usw. (false default)
	    locale 		= DE,		% lacale= DE, uses german
	    per			= slash,
%	    loctolang		={
%				  UK:english, 
%				  DE:ngerman
%		      		 },		% loctolang={USA:USenglish,DE:ngerman}, if language changed in document
	    detect-all,				% detect-all : uses the text options not the math mode to set the letters/numbers
	   ]{siunitx}


\usepackage[Algorithmus]{algorithm} 			% deutsche notation im titel

\usepackage[%
%	    leqno,				% Nummern linksbuendig
%	    reqno,				% Nummern rechtsbuendig (Standard)
%	    fleqn,				% Gleichungen linksbuendig statt zentriert
	   ]{amsmath}
\usepackage{amsfonts}				% Um Zahlenraeume richtig darzustellen
\usepackage{mathtools}


% Betragsstriche über \abs, Doppelbetragsstriche über \norm
\DeclarePairedDelimiter\abs{\lvert}{\rvert}%
\DeclarePairedDelimiter\norm{\lVert}{\rVert}%

\usepackage[%
	    thinlines,				% dünne linien
%	    thicklines,				% dicke linien
	   ]{easybmat}				% For Matrices with dottet lines between fields, horizontal or vertical
%\usepackage{MnSymbol}				% Zusaetzliche Zeichen
\usepackage{trsym}				% Fuer Laplace-Fourier-Symbole
\usepackage{mathrsfs}				% Fuer Matheschrift
\usepackage{xfrac}
\usepackage{nicefrac}
\usepackage{esint}
%\usepackage{exscale} % lässt Klammern in Gleichungen verschwinden!!
%\DeclareMathSizes{10.95}{30}{15}{15} % 10.95 für 11pt in documentclass

\def\mathunderline#1#2{\color{#1}\underline{{\color{black}#2}}\color{black}}

\usepackage{soul}

\usepackage[%				
%	    group-separator 	= {.}, 		% group-seperator={,} -> 1,345,234.23
%   	    round-mode 		= places,		% round-mode= places(2digits), figures(1digit)
%	    round-precision 	= 3,		% round-precision= x , xdigits are rounded to
	    binary-units 	= true,		% Laden von \byte \bit, \kibi usw. (false default)
	    locale 		= DE,		% lacale= DE, uses german
	    per			= slash,
	    alsoload            = binary,
%	    loctolang		={
%				  UK:english, 
%				  DE:ngerman
%		      		 },		% loctolang={USA:USenglish,DE:ngerman}, if language changed in document
	    detect-all				% detect-all : uses the text options not the math mode to set the letters/numbers
	   ]{siunitx}

\usepackage{multirow}				% Fuer Mehrzeilige Zellen
\usepackage{multicol}
%\usepackage{array}					% Festlegen von Breiten, Präfixe, Suffixe
% see also package siunitx - damit Spalten mit Zahlen auf bestimmte Länge ausgerichtet werden
\usepackage{tabularx}				% Zum festlegen der Gesamtbreite der Tabelle & verwenden von X fuer 
									% variable Spaltenbreite
%\usepackage{longtable}				% Tabelle ueber mehr als eine Seite
\usepackage{ltxtable}				% longtable + tabularx eigenschaften

%\usepackage{rotating}				% Querformat der Tabelle
									% !!! Vorsicht !!!
									% tablecaptionabove wird nicht ausgewertet,
									% Einfuegen von \vskip\abovecaptionskip --> Nur unter KOMA Script
\usepackage{ctable}					% For tables with footnote under the table directly


\setlength{\arraycolsep}{0.4pt}
\delimitershortfall=0pt % Höhereduzierung der Klammern von Matrizen gegenüber dem Inhalt

\usepackage{tabu} % fuer farbige Zeilen in Tabellen
