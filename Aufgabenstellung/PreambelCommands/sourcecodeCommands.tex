%%%%%%%%%%%%%%%%%%%%%%%%%%%%%%%%%%%%%%%%%%%%%%%%%%%%%%%%%%%%%%%%%%%%%%%%%%%%%%%
%  Author     : Thorben Schuethe        Date    : 10.10.2011                  %
%  Filename   : sourceCodeCommands.tex Version : 1.7                          %
%-----------------------------------------------------------------------------%
%  Changelog  :                                                               %
%  10.09.2011 : build date, Only Matlab and C-Code supported               1.0%
%  14.09.2011 : Added the check of color package loaded                    1.1%
%  10.10.2011 : add java                                                   1.2%
%  02.11.2011 : add shell                                                  1.3%
%  22.12.2011 : add xml                                                    1.4%
%  26.12.2011 : added commands defcommunicate and defstate                v1.5%
%  23.08.2012 : Add Pseudo Code and options printLstOpt                   v1.6%
%  11.10.2012 : Add cpp und ändern der Java umgebung                      v1.7%
%%%%%%%%%%%%%%%%%%%%%%%%%%%%%%%%%%%%%%%%%%%%%%%%%%%%%%%%%%%%%%%%%%%%%%%%%%%%%%%
% define new commands to check if packages are loadedd or not
\makeatletter
\providecommand{\IfPackageLoaded}[2]{\@ifpackageloaded{#1}{#2}{}}
\providecommand{\IfPackageNotLoaded}[2]{\@ifpackageloaded{#1}{}{#2}}
\providecommand{\IfElsePackageLoaded}[3]{\@ifpackageloaded{#1}{#2}{#3}}
\makeatother
%%%%%%%%%%%%%%%%%%%%%%%%%%%%%%%%%%%%%%%%
% global Definitions / Packages        %
%%%%%%%%%%%%%%%%%%%%%%%%%%%%%%%%%%%%%%%%
\IfPackageLoaded{color}{%
\definecolor{commMatlab}{RGB}{34,139,34}
\definecolor{commandJAVA}{RGB}{204,0,102}
\definecolor{shellcolor}{RGB}{230,230,230}
}
\newcommand{\printLstOpt}{htp}						% Option für die Position
\newcommand{\lineSkipSpace}{-3pt}					% Option um Zeilenabstand zu vergrößern/verringern
%%%%%%%%%%%%%%%%%%%%%%%%%%%%%%%%%%%%%%%%

%%%%%%%%%%%%%%%%%%%%%%%%%%%%%%%%%%%%%%%%
% global Definitions                   %
%%%%%%%%%%%%%%%%%%%%%%%%%%%%%%%%%%%%%%%%
\IfPackageLoaded{listings}{%
	\renewcommand{\lstlistlistingname}{Quellcodeverzeichnis} %LST Verzeichnisname
	\renewcommand{\lstlistingname}{Quellcode}				 %LST NAME
%%%%%%%%%%%%%%%%%%%%%%%%%%%%%%%%%%%%%%%%
% All for Matlab Source-Codes          %
%%%%%%%%%%%%%%%%%%%%%%%%%%%%%%%%%%%%%%%%
% Matlab environment, for in text source code
\IfElsePackageLoaded{color}{%
	\lstnewenvironment{Matlab}[1][]				%load with color options
	{\lstset{language			= Matlab,
		     numberstyle		= \tiny,
		     numbersep			= 1em, 
		     numbers			= left,
		     breaklines 		= true,
		     breakautoindent 	= true,
		     inputencoding		= latin1,
		     basicstyle			= \ttfamily, 
		     tabsize			= 4,
		     stepnumber			= 1,
		     backgroundcolor	= \color{white},
		     keywordstyle		= \color{blue},
		     commentstyle		= \color{commMatlab},
		     frame				= single,
		     framexleftmargin	= 2.7em,				% vom quelltext zum rahmen
		     xleftmargin		= 3em,
		     float 				= \printLstOpt ,
		     #1}
	}{}
	}
	{\lstnewenvironment{Matlab}[1][]				% load without color options
	{\lstset{language			= Matlab,
		     numberstyle		= \tiny,
		     numbersep			= 1em, 
		     numbers			= left,
		     breaklines 		= true,
		     breakautoindent 	= true,
		     inputencoding		= latin1,
		     basicstyle			= \ttfamily,
		     tabsize			= 4,
		     stepnumber			= 1,
		     frame				= single,
		     framexleftmargin	= 2.7em,				% vom quelltext zum rahmen
		     xleftmargin		= 3em,
		     float 				= \printLstOpt ,
		     #1}
	}{}}
% for including matlab files directly
\IfElsePackageLoaded{color}{%
	\newcommand{\matlabfile}[1]{\lstset{%
			 language			= Matlab,
		     numberstyle		= \tiny,
		     numbersep			= 1em, 
		     numbers			= left,
		     breaklines 		= true,
		     breakautoindent 	= true,
		     inputencoding		= latin1,
		     basicstyle			= \ttfamily,
		     tabsize			= 4,
		     stepnumber			= 2, 
		     backgroundcolor	= \color{white},
		     keywordstyle		= \color{blue},
		     commentstyle		= \color{commMatlab},
		     frame				= single,
		     framexleftmargin	= 2.7em,				% vom quelltext zum rahmen
		     xleftmargin		= 3em,
		     float 				= \printLstOpt ,}%
		     \lstinputlisting{#1}}
}
{\newcommand{\matlabfile}[1]{\lstset{%
			 language			= Matlab,
		     numberstyle		= \tiny,
		     numbersep			= 1em, 
		     numbers			= left,
		     breaklines 		= true,
		     breakautoindent 	= true,
		     inputencoding		= latin1,
		     basicstyle			= \ttfamily,
		     tabsize			= 4,
		     stepnumber			= 2, 
		     frame				= single,
		     framexleftmargin	= 2.7em,				% vom quelltext zum rahmen
		     xleftmargin		= 3em,
		     float 				= \printLstOpt ,}%
		     \lstinputlisting{#1}}
}
%%%%%%%%%%%%%%%%%%%%%%%%%%%%%%%%%%%%%%%%
% End Matlab Source-Codes              %
%%%%%%%%%%%%%%%%%%%%%%%%%%%%%%%%%%%%%%%%

%%%%%%%%%%%%%%%%%%%%%%%%%%%%%%%%%%%%%%%%
% All for Ansi-C Source-Codes          %
%%%%%%%%%%%%%%%%%%%%%%%%%%%%%%%%%%%%%%%%
\IfElsePackageLoaded{color}{%
	\newcommand{\cfile}[1]{\lstset{%
			 language			= C,
		     numberstyle		= \tiny,			% kleine zahlen
		     numbersep			= 1em, 			% abstand der zahlen 5em
		     numbers			= left,				% nummerierung links
		     breaklines 		= true,				% erlaubt automatische zeilenumbrueche des codes
		     breakautoindent 	= true,				% einschub nach umbruch
		     inputencoding		= latin1,			% quellcode codierung
		     basicstyle			= \ttfamily,		% lade truetype family schrift
		     tabsize			= 4,				% tabsize im quellcode = 4
		     stepnumber			= 2,				% nummerierung jede 2te zahl
		     backgroundcolor	= \color{white},
		     keywordstyle		= \color{blue},
		     commentstyle		= \color{commMatlab},
		     frame				= single,			% zeichne rahmen um den quellcode
		     framexleftmargin	= 2.7em,				% vom quelltext zum rahmen
		     xleftmargin		= 3em,
		     float 				= \printLstOpt ,}%				% rahmen 7mm links von den zahlen zeichnen
		     \lstinputlisting{#1}}
}
{\newcommand{\cfile}[1]{\lstset{%
			 language			= C,
		     numberstyle		= \tiny,			% kleine zahlen
		     numbersep			= 1em, 			% abstand der zahlen 5em
		     numbers			= left,				% nummerierung links
		     breaklines 		= true,				% erlaubt automatische zeilenumbrueche des codes
		     breakautoindent 	= true,				% einschub nach umbruch
		     inputencoding		= latin1,			% quellcode codierung
		     basicstyle			= \ttfamily,		% lade truetype family schrift
		     tabsize			= 4,				% tabsize im quellcode = 4
		     stepnumber			= 2,				% nummerierung jede 2te zahl
		     frame				= single,			% zeichne rahmen um den quellcode
		     framexleftmargin	= 2.7em,				% vom quelltext zum rahmen
		     xleftmargin		= 3em,
		     float 				= \printLstOpt ,}%				% rahmen 7mm links von den zahlen zeichnen
		     \lstinputlisting{#1}}
}
\IfElsePackageLoaded{color}{%
\newcommand{\setC}{\lstset{%
%			 morekeywords		={include , define},
			 language			= C++,
		     numberstyle		= \tiny,
		     numberfirstline	= true,
		     numbersep			= 1em, 
		     numbers			= left,
		     breaklines 		= true,
		     breakautoindent 	= true,
		     inputencoding		= latin1,
		     showstringspaces	= false,			% default is true - shows blank spaces
		     basicstyle			= \ttfamily,
		     tabsize			= 2,
		     stepnumber			= 1,
		     backgroundcolor	= \color{white},
		     keywordstyle		= \color{blue},
		     commentstyle		= \color{commMatlab},
		     frame				= single,
		     framexleftmargin	= 2.7em,				% vom quelltext zum rahmen
		     xleftmargin		= 3em,
		     float 				= \printLstOpt ,
		     }}
}
{\newcommand{\setC}{\lstset{%
			 language			= C,
		     numberstyle		= \tiny,
		     numberfirstline	= true,
		     numbersep			= 1em, 
		     numbers			= left,
		     breaklines 		= true,
		     breakautoindent 	= true,
		     inputencoding		= latin1,
		     showstringspaces	= false,			% default is true - shows blank spaces
		     basicstyle			= \ttfamily,
		     tabsize			= 4,
		     stepnumber			= 1,
		     frame				= single,
		     framexleftmargin	= 2.7em,				% vom quelltext zum rahmen
		     xleftmargin		= 3em,
		     float 				= \printLstOpt ,}}
}

%%%%%%%%%%%%%%%%%%%%%%%%%%%%%%%%%%%%%%%%
% All for c++ Source-Codes             %
%%%%%%%%%%%%%%%%%%%%%%%%%%%%%%%%%%%%%%%%
\IfElsePackageLoaded{color}{%
	\lstnewenvironment{cpplus}[1][]				%load with color options
	{\lstset{language			= C++,
		     numberstyle		= \tiny,
		     numbersep			= 1em, 
		     numbers			= left,
		     breaklines 		= true,
		     breakautoindent 	= true,
		     inputencoding		= latin1,
		     basicstyle			= \ttfamily, 
		     tabsize			= 2,
		     stepnumber			= 1,
		     backgroundcolor	= \color{white},
		     keywordstyle		= \color{blue},
		     commentstyle		= \color{commMatlab},
		     frame				= single,
		     framexleftmargin	= 2.7em,				% vom quelltext zum rahmen
		     xleftmargin		= 3em,
		     float 				= \printLstOpt ,
		     #1}
	}{}
	}
	{\lstnewenvironment{cpplus}[1][]				% load without color options
	{\lstset{language			= C++,
		     numberstyle		= \tiny,
		     numbersep			= 1em, 
		     numbers			= left,
		     breaklines 		= true,
		     breakautoindent 	= true,
		     inputencoding		= latin1,
		     basicstyle			= \ttfamily,
		     tabsize			= 2,
		     stepnumber			= 1,
		     frame				= single,
		     framexleftmargin	= 2.7em,				% vom quelltext zum rahmen
		     xleftmargin		= 3em,
		     float 				= \printLstOpt ,
		     #1}
	}{}}
	
\IfElsePackageLoaded{color}{%
	\newcommand{\cppfile}[2]{%
		     \lstinputlisting[language			= C++,
		     numberstyle		= \tiny,
		     numbersep			= 1em, 
		     numbers			= left,
		     breaklines 		= true,
		     breakautoindent 	= true,
		     inputencoding		= latin1,
		     basicstyle			= \ttfamily, 
		     tabsize			= 2,
		     stepnumber			= 1,
		     backgroundcolor	= \color{white},
		     keywordstyle		= \color{blue},
		     commentstyle		= \color{commMatlab},
		     frame				= single,
		     framexleftmargin	= 2.7em,				% vom quelltext zum rahmen
		     xleftmargin		= 3em,
		     float 				= \printLstOpt ,#1]{#2}}
}
{\newcommand{\cppfile}[3]{%
		     \lstinputlisting[language			= C++,
		     numberstyle		= \tiny,
		     numbersep			= 1em, 
		     numbers			= left,
		     breaklines 		= true,
		     breakautoindent 	= true,
		     inputencoding		= latin1,
		     basicstyle			= \ttfamily,
		     tabsize			= 2,
		     stepnumber			= 1,
		     frame				= single,
		     framexleftmargin	= 2.7em,				% vom quelltext zum rahmen
		     xleftmargin		= 3em,
		     float 				= \printLstOpt ,#1]{#2}
}}	
%%%%%%%%%%%%%%%%%%%%%%%%%%%%%%%%%%%%%%%%
% All for JAVA Source-Codes          %
%%%%%%%%%%%%%%%%%%%%%%%%%%%%%%%%%%%%%%%%
\IfElsePackageLoaded{color}{%
	\newcommand{\javafile}[1]{\lstset{%
			 language			= Java,
		     numberstyle		= \tiny,			% kleine zahlen
		     numbersep			= 1em, 			% abstand der zahlen 5em
		     numbers			= left,				% nummerierung links
		     breaklines 		= true,				% erlaubt automatische zeilenumbrueche des codes
		     breakautoindent 	= true,				% einschub nach umbruch
		     inputencoding		= latin1,			% quellcode codierung
		     showstringspaces	= false,			% default is true - shows blank spaces
		     basicstyle			= \ttfamily,		% lade truetype family schrift
		     tabsize			= 4,				% tabsize im quellcode = 4
		     stepnumber			= 2,				% nummerierung jede 2te zahl
		     backgroundcolor	= \color{white},
		     keywordstyle		= \color{commandJAVA},
		     commentstyle		= \color{commMatlab},
		     frame				= single,			% zeichne rahmen um den quellcode
		     framexleftmargin	= 2.7em,				% vom quelltext zum rahmen
		     xleftmargin		= 3em,
		     float 				= \printLstOpt ,}%			% rahmen 7mm links von den zahlen zeichnen
		     \lstinputlisting{#1}}
}
{\newcommand{\javafile}[1]{\lstset{%
			 language			= Java,
		     numberstyle		= \tiny,			% kleine zahlen
		     numbersep			= 1em, 			% abstand der zahlen 5em
		     numbers			= left,				% nummerierung links
		     breaklines 		= true,				% erlaubt automatische zeilenumbrueche des codes
		     breakautoindent 	= true,				% einschub nach umbruch
		     inputencoding		= latin1,			% quellcode codierung
   		     showstringspaces	= false,			% default is true - shows blank spaces
		     basicstyle			= \ttfamily,		% lade truetype family schrift
		     tabsize			= 4,				% tabsize im quellcode = 4
		     stepnumber			= 2,				% nummerierung jede 2te zahl
		     frame				= single,			% zeichne rahmen um den quellcode
		     framexleftmargin	= 2.7em,				% vom quelltext zum rahmen
		     xleftmargin		= 3em,
		     float 				= \printLstOpt ,}%				% rahmen 7mm links von den zahlen zeichnen
		     \lstinputlisting{#1}}
}
% JAVA Environment to include SourceCode in Text
\IfElsePackageLoaded{color}{%
	\lstnewenvironment{Java}[1][]				%load with color options
	{\lstset{language			= Java,
		     numberstyle		= \tiny,
		     numbersep			= 1em, 
		     numbers			= left,
		     breaklines 		= true,
		     breakautoindent 	= true,
		     inputencoding		= latin1,
		     showstringspaces	= false,			% default is true - shows blank spaces
		     basicstyle			= \ttfamily,
		     tabsize			= 2,
		     stepnumber			= 2,
		     backgroundcolor	= \color{white},
		     keywordstyle		= \color{commandJAVA},
		     commentstyle		= \color{commMatlab},
		     frame				= single,
		     framexleftmargin	= 2.7em,				% vom quelltext zum rahmen
		     xleftmargin		= 3em,
		     float 				= \printLstOpt ,
		     #1}
	}{}
	}
	{\lstnewenvironment{Java}[1][]				% load without color options
	{\lstset{language			= Java,
		     numberstyle		= \tiny,
		     numbersep			= 1.5em, 
		     numbers			= left,
		     breaklines 		= true,
		     breakautoindent 	= true,
		     inputencoding		= latin1,
		     showstringspaces	= false,			% default is true - shows blank spaces
		     basicstyle			= \ttfamily,
		     tabsize			= 2,
		     stepnumber			= 2,
		     frame				= single,
		     framexleftmargin	= 2.7em,				% vom quelltext zum rahmen
		     xleftmargin		= 3em,
		     float 				= \printLstOpt ,
		     #1}
	}{}}
	
	\newcommand{\setJava}{\lstset{%
			 language			= Java,
		     numberstyle		= \tiny,
		     numberfirstline	= true,
		     numbersep			= 1.5em, 
		     numbers			= left,
		     breaklines 		= true,
		     breakautoindent 	= true,
		     inputencoding		= latin1,
		     showstringspaces	= false,			% default is true - shows blank spaces
		     basicstyle			= \ttfamily,
		     tabsize			= 4,
		     stepnumber			= 1,
		     backgroundcolor	= \color{white},
		     keywordstyle		= \color{commandJAVA},
		     commentstyle		= \color{commMatlab},
		     frame				= single,
		     framexleftmargin	= 2.7em,				% vom quelltext zum rahmen
		     xleftmargin		= 3em,
		     float 				= \printLstOpt}}


%%%%%%%%%%%%%%%%%%%%%%%%%%%%%%%%%%%%%%%%
% All for SHELL Commands               %
%%%%%%%%%%%%%%%%%%%%%%%%%%%%%%%%%%%%%%%%
\lstnewenvironment{Shell}[1][]{
   \lstset{#1}   
   \lstset{
	  language			= bash,
	  numbers			= none,
	  breaklines 		= true,
      breakautoindent 	= true,
%	  inputencoding		= latin1,
	  showstringspaces	= false,			% default is true - shows blank spaces
      basicstyle		= \ttfamily,
      tabsize			= 2,
      backgroundcolor	= \color{shellcolor},
      keywordstyle		= \color{black},
	  commentstyle		= \color{black},
      frame				= none,
      framexleftmargin	= 0mm,
      float 			= \printLstOpt
	  }
}{}

%%%%%%%%%%%%%%%%%%%%%%%%%%%%%%%%%%%%%%%%
% FOR PSEUDO CODE                      %
%%%%%%%%%%%%%%%%%%%%%%%%%%%%%%%%%%%%%%%%
\lstdefinelanguage{pseu}{%
	morekeywords={begin, for, do, od, proc, if, fi, end, then, exit, step, where, while, elseif, elif, else},
	sensitive=false,
	morecomment=[l]{//},
	morecomment=[s]{/*}{*/},
	morestring=[b]",%
}

\lstnewenvironment{Pseudo}[1][]{%
   \lstset{%
	  language			= pseu,
	  numberstyle		= \tiny,
      numberfirstline	= true,
      numbersep			= 1.5em, 
      numbers			= left,
      stepnumber        = 2,
	  breaklines 		= true,
      breakautoindent 	= true,
	  showstringspaces	= false,			% default is true - shows blank spaces
      basicstyle		= \ttfamily,
      tabsize			= 2,
      backgroundcolor	= \color{shellcolor},
      keywordstyle		= \color{black}\bfseries,
	  commentstyle		= \color{black},
      frame				= none,
      framexleftmargin	= 2.7em,				% vom quelltext zum rahmen
      xleftmargin		= 3em,
	  float 			= \printLstOpt ,
	  lineskip          = \lineSkipSpace ,	% space extra to normal line skip which can be +- ...pt
	  #1,									% Option for the user
	  }
}{}

%%%%%%%%%%%%%%%%%%%%%%%%%%%%%%%%%%%%%%%%
% All for XML-Files                    %
%%%%%%%%%%%%%%%%%%%%%%%%%%%%%%%%%%%%%%%%
	\IfElsePackageLoaded{color}{%
	\newcommand{\xmlfile}[1]{\lstset{%
			 language			= XML,
		     numberstyle		= \tiny,			% kleine zahlen
		     numbersep			= 1em, 			% abstand der zahlen 5em
		     numbers			= left,				% nummerierung links
		     breaklines 		= true,				% erlaubt automatische zeilenumbrueche des codes
		     breakautoindent 	= true,				% einschub nach umbruch
		     inputencoding		= latin1,			% quellcode codierung
		     showstringspaces	= false,			% default is true - shows blank spaces
		     basicstyle			= \ttfamily,		% lade truetype family schrift
		     tabsize			= 4,				% tabsize im quellcode = 4
		     stepnumber			= 2,				% nummerierung jede 2te zahl
		     backgroundcolor	= \color{white},
		     keywordstyle		= \color{commandJAVA},
		     commentstyle		= \color{commMatlab},
		     frame				= single,			% zeichne rahmen um den quellcode
		     framexleftmargin	= 2.7em,				% vom quelltext zum rahmen
		     xleftmargin		= 3em,
		     float 				= \printLstOpt ,}%				% rahmen 7mm links von den zahlen zeichnen
		     \lstinputlisting{#1}}
}
{\newcommand{\xmlfile}[1]{\lstset{%
			 language			= XML,
		     numberstyle		= \tiny,			% kleine zahlen
		     numbersep			= 1em, 			% abstand der zahlen 5em
		     numbers			= left,				% nummerierung links
		     breaklines 		= true,				% erlaubt automatische zeilenumbrueche des codes
		     breakautoindent 	= true,				% einschub nach umbruch
		     inputencoding		= latin1,			% quellcode codierung
   		     showstringspaces	= false,			% default is true - shows blank spaces
		     basicstyle			= \ttfamily,		% lade truetype family schrift
		     tabsize			= 4,				% tabsize im quellcode = 4
		     stepnumber			= 2,				% nummerierung jede 2te zahl
		     frame				= single,			% zeichne rahmen um den quellcode
		     framexleftmargin	= 2.7em,				% vom quelltext zum rahmen
		     xleftmargin		= 3em,
		     float 				= \printLstOpt ,}%				% rahmen 7mm links von den zahlen zeichnen
		     \lstinputlisting{#1}}
}     

\lstnewenvironment{Xml}[1][]{				%load with color options
	\lstset{language			= XML,
		     numberstyle		= \tiny,			% kleine zahlen
		     numbersep			= 1em, 				% abstand der zahlen 5em
		     numbers			= left,				% nummerierung links
		     breaklines 		= true,				% erlaubt automatische zeilenumbrueche des codes
		     breakautoindent 	= true,				% einschub nach umbruch
		    % inputencoding		= utf8,				% quellcode codierung
   		     showstringspaces	= false,			% default is true - shows blank spaces
		     basicstyle			= \ttfamily,		% lade truetype family schrift
		     tabsize			= 4,				% tabsize im quellcode = 4
		     stepnumber			= 2,				% nummerierung jede 2te zahl
		     frame				= single,			% zeichne rahmen um den quellcode
		     framexleftmargin	= 2.7em,				% vom quelltext zum rahmen
		     xleftmargin		= 3em,
		     float 				= \printLstOpt ,
		     #1}
}{}
}