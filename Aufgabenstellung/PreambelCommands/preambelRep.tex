%%%%%%%%%%%%%%%%%%%%%%%%%%%%%%%%%%%%%%%%%%%%%%%%%%%%%%%%%%%%%%%%%%%%%%%%%%%%%%%
%  Author     : Thorben Schuethe        Date    : 28.12.2011                  %
%  Filename   : preambelRep.tex        Version : 2.12                         %
%  Information: This file contains all build information which must be loaded %
%               before \begin{document}. All packages needed for writing are  %
%               loaded in this preambel. File will be safed and used with     %
%               UTF-8 Format                                                  %
%-----------------------------------------------------------------------------%
%  Changelog  :                                                               %
%  10.09.2011 : build date, Koma-Script packages for report                   %
%  14.09.2011 : add Kopfzeilen, Layoutparameter                               %
%  03.10.2011 : schriften laden, pdfhyperref infos, crop eingebunden          %
%  07.10.2011 : documentclass erweitert mit oneside, onecolumn,draft          %
%  10.10.2011 : aenderung der bibtex encodierung auf utf8                     %
%  02.11.2011 : epstopdf eingebunden                                          %
%  26.11.2011 : pdf pages eingebunden, draft auskommentiert                   %
%  28.12.2011 : added acronym, tested nomenclature, added multicol            %
%  23.08.2012 : scrhack um warnungen vom listing zu entfernen                 %
%  03.09.2012 : siunitx - option binary geladen + glossary hinzugefügt        %
%  04.09.2012 : add glossaries - for better handling of acronym and symbols   %
%  16.09.2012 : Mathepakete fuer andere Symbole geladen                       %
%  13.10.2012 : ctable fuer Tabellen mit Fußnote                              %
%  20.11.2012 : Option to epstopdf package                                    %
%  21.11.2012 : Make living column title aktiv                                %
%%%%%%%%%%%%%%%%%%%%%%%%%%%%%%%%%%%%%%%%%%%%%%%%%%%%%%%%%%%%%%%%%%%%%%%%%%%%%%%

% load the document class KOMA-Script Report
\documentclass[	a4paper,			% Papierformat waehlen
				11pt,				% Schriftgroesse
%				draft,				% Randueberschreitungen werden schwarz markiert
				DIV13,				% Unterteilung des Blatts in DIVxx - xx Teile
									% Je kleiner die Zahl umso groesser die Raender
									% Um Optimale Anpassung zu erhalten : DIVcalc + \typearea...
				BCOR5mm,			% Bindekorrektur BCORXmm - Xmm breit
% 		KEIN ZWEITER PUNKT IN DER NUMMERIERUNG DER ÜBERSCHRIFTET 2.2 statt 2.2.	
				pointlessnumbers,			
%      KOPF/FUSSZEILE
%				headinclude,		% Head gehoert mit zum Textblock/auskommentiert im Seitenrand
				headsepline,		% Trennlinie zwischen Kopf und Text, schaltet automatisch 
%									 headinclude mit ein
%				footinclude,		% wie Head nur Foot
%				footsepline,		% wie headsepline
%      LAYOUTPARAMETER
%				twocolumn,			% Zweispaltige Texte
				onecolumn,			% Einspaltig
%				twoside,			% Zweiseitiger Druck re -li sind gespiegelt vom Satzspiegel
				oneside,			% Einseitig
%				openright,			% Sollte bei zweiseitigem Druck eignestellt werden
%				cleardoublestandard	% Alles fuer 2Seitige mit openright: Kolumnentitel linke Seite
%				cleardoubleplain	% nur Seitenzahlen linke Seite
%				cleardoubleempty	% linke Seite ganz leer
%				chapterprefix,		% Kapitel wir zu beginn eines Kapitels ausgegeben
%				appendixprefix,		% Anhang wird vor Kapiteln im Anhang ausgegeben
				headings=normal,	% kleine ueberschriften, oder small-, oder big-headings.
									% Standard ist Big
%		TABELLENEIGENSCHAFTEN
				captions=tableheading,	% Um richtigen Abstand fuer Ueberschrift zu erhalten, 
									% Ueberschrift wird genommen
% 		VERZEICHNISEINSTELLUNGEN
%				tocleft,			% Setzt das Inhaltverzeichnis nicht eingerueckt nach links
%				liststotoc,			% Abb.- und Tab.Verzeichnisse ins Inhaltsverzeichnis
%				liststotocnumbered, % wie liststotoc nur mit Gliederungspunktangabe
%				bibtotoc,			% Literaturverzeichnis ins Inhaltverzeichnis
%				bibtotocnumbered,	% s.o. nur mit Gliederungsebene
%				idxtotoc,			% Eintrag ins Inhaltsverzeichnis fue Index
%				openbib,			% Veraendert aussehen des Lit.-Verzeichnisses
%		Language
%				ngerman,			% to pass as standard to all packages
						]{scrreprt}
%%%%%%%%%%%%%%%%%%%%%%%%%%%%%%%%%%%%%%%
% Schnittmarken falls notwendig       %
%%%%%%%%%%%%%%%%%%%%%%%%%%%%%%%%%%%%%%%
%\usepackage[% Papierformate auf denen gedruckt werden soll
%			a0,b0,
%			a1,b1,
%			a2,b2,
%			a3,b3,
%			a4,b4,
%			a5,b5,
%			a6,b6,
%			letter,
%			legal,
%			executive,
%			center,		% centrierter druck
%			landscape,	% querformat
%			]{crop}

%%%%%%%%%%%%%%%%%%%%%%%%%%%%%%%%%%%%%%%
% Loading writingpackages             %
%%%%%%%%%%%%%%%%%%%%%%%%%%%%%%%%%%%%%%%

% load packages for european, espacially german users
\usepackage[T1]{fontenc}			% Schriften fuer Europaeische Zeichen passend Kodiert
\usepackage[utf8]{inputenc}         % Eingabe von Umlauten, ss usw. - 
									% utf8: kann nicht jeder Editor speicher, aber plattformuebergreifend
									% latin1: Unix, VMS, Windows
									% ansinew: Windows
									% latin9: wie latin1, jedoch mit Eurozeichen
\usepackage[english, ngerman]{babel}% Passt Konventionen an Sprache an
									% ngerman: Neue Deutsche Rechtschreibung
									% oder z.B. UKenglish, USenglish, canadian, german, austrian, 	
									% naustrian
									
% Schriftfamilien Laden
\usepackage{lmodern}				% Laedt die Schriftart Latin-Modern fuer Text
%\usepackage[scaled=0.66]{luximono}	% Schrift fuer Typewriter - Quellcode ~8pt
									
%\usepackage[a4paper]{geometry}		% Anpassung von Seitenraender per Hand
%\geometry{	left=3cm, top=2cm, 
%			right=2cm, bottom=3cm} 	% Wenn Zweiseitig-> left->inner : right->outer
									% Weitere Moeglichkeiten: hight - Texthoehe, width - Textbreite

% FUSS UND KOPFZEILENANPASSUNG
%\pagestyle{plain}					% empty: Keine Kopf/Fusszeile
									% plain: Seitenzahl am Fussende
									% headings: aktiviert lebende Kolumnentitel (Kapitel im Kopf)
									% myheadings : eigene Kopf- fußzeilen
									% fancy: Erlaubt die Verwendung der in dem Paket "fancyhdr" 
									% definierten Befehle zur Erstellung eigener Kopf- und Fußzeilen
\usepackage[%
			automark				% takes the chapter variant
			]{scrpage2}				% Fuer eigene Kopf/Fusszeilen, ermoeglicht 
\pagestyle{scrheadings} %, scrpalin}
\clearscrheadings					% clear old header style
\clearscrplain						% clear plain header style
\clearscrheadfoot					% clear foot style
%\automark[section]{chapter}
\ohead{\pagemark}					% beide Außenseiten des Headers --Seitenzahl
\ihead{\headmark}					% beide Innenseiten Headers -- Chapter
%\lehead{\leftmark}
%\lohead{\rightmark}
%\ofoot[\pagemark]{}					% auf plain seiten Seitenzahl außen
									
%\usepackage{setspace}				% Fuer Abstaende im Dokument
%\onehalfspacing					% aendert Zeilenabstand auf 1 1/2
									
%\usepackage{microtype}				% optischer Randausgleich in PDF's
									% ungeeignet fuer internettexte o.ae.	
									
\usepackage[						% To set Text at a specific 
			absolute,				% absolute - absolute position on the page
			overlay,				% overlay - if using absolute option text is placed below other things
%			showboxes,				% showboxes - shows boxes around the text
%			noshowtext,				% noshowtext - just show the box if it is on
%			verbose,				% verbose - package writing things to output like calculations
%			quiet,					%quiet turns this off : verbose = default
			]{textpos}		

									
\usepackage{bbding}					% to get cross and check symbol
									
									
%%%%%%%%%%%%%%%%%%%%%%%%%%%%%%%%%%%%%%%
% Graphic/Color Packages              %
%%%%%%%%%%%%%%%%%%%%%%%%%%%%%%%%%%%%%%%
\usepackage{color}					% to use colors, espacially for source codes
\usepackage[%
			%table					% Zum automatischen Laden des Pakets colortbl - fuer Tabellen
			]{xcolor}				% for Hyperref package, that one can say red than rgb values
\usepackage{graphicx}				% laden der Graphic Optionen
\usepackage{subfigure}				% Für Untergrafiken, wenn eine eigene Bildunterschrift erwuenscht
\usepackage{pdfpages}				% Zum Laden und einbinden von PDF-Seiten
\usepackage[%
			update					% if pdf conversion is older than eps file, new conversion is done
			]{epstopdf}				% eps to pdf needs for pdflatex - after graphicx
\usepackage{framed}					% for use of newtheorem with a coloured box

\setlength{\headheight}{1.5cm}
\setlength{\voffset}{0.5cm}
\usepackage{geometry}
\geometry{
  left=1.5cm,
  right=1.5cm,
  top=2cm,
  bottom=2.8cm,
  bindingoffset=0mm
}
%%%%%%%%%%%%%%%%%%%%%%%%%%%%%%%%%%%%%%%
% Math Packages                       %
%%%%%%%%%%%%%%%%%%%%%%%%%%%%%%%%%%%%%%%
\usepackage[%
%			leqno,					% Nummern linksbuendig
%			reqno,					% Nummern rechtsbuendig (Standard)
%			fleqn,					% Gleichungen linksbuendig statt zentriert
			]{amsmath}
\usepackage{amsfonts}				% Um Zahlenraeume richtig darzustellen
\usepackage{MnSymbol}				% Zusaetzliche Zeichen
\usepackage{mathrsfs}				% Fuer Matheschrift
%\usepackage{icomma}					% Bei deutscher Schreibweise, damit keine
									% leerzeichen zwischen , und Zahl
\usepackage{nicefrac}				% Fuer bessere Darstellung von Bruechen im 
									% Fließtext
\usepackage{trsym}					% Fuer Laplace-Fourier-Symbole
\usepackage[%
			right,					% Place of the Euro-Sign
			]{eurosym}
\usepackage[%
			thinlines,				% dünne linien
%			thicklines,				% dicke linien
			]{easybmat}				% For Matrices with dottet lines between fields, horizontal or vertical
%%%%%%%%%%%%%%%%%%%%%%%%%%%%%%%%%%%%%%%
%   Berechnung der Seitenraender      %
%%%%%%%%%%%%%%%%%%%%%%%%%%%%%%%%%%%%%%%
% Fuer optimale Anpassung des Satzspiegels
%\typearea[current]{calc}				% Bestimmt Rasterzahl aus Schriftgroesse und Anzahl der 		
									% Zeichen/Woerter pro Zeile
%%%%%%%%%%%%%%%%%%%%%%%%%%%%%%%%%%%%%%%
% Verzeichniss Packages               %
%%%%%%%%%%%%%%%%%%%%%%%%%%%%%%%%%%%%%%%
\usepackage[babel,
%			german=guillemets,  	% franz. <<  >>
			german= quotes,			% deutsche Anfuehrungszeichen "
			]{csquotes}				% Laden der richtigen Anfuehrungszeicehn fuer deutsche sprache
									% = quotes fuer englische sprache
\usepackage[backend=biber,%
%			style=numeric-comp,%	% Zitiert durch Nummern - also [1], [siehe 1-3]
			style=ieee,
%			style=numeric,%			% Zitiert durch Nummern - also [1], [siehe 2,1,3]
%			sorting,nty,%			% Sortieren nach Name,Titel,Jahr - diese Option brauch nicht
									% geladen werden, da dies der automatische zitierstyl selbst-
									% staendig uebernimmt
			defernumbers=true,      % to set different numeric types -e.g. A1...A2 B1...B2
			sortcase=false,%		% false- keine unterscheidung zwischen gross und kleinschrift
			autolang=other,%		% jeder eintrag in seiner sprache
			bibencoding=utf8,%
			]{biblatex}				% um 8-bit zeichen zu erkennen, fuer umlaute

% 	INDEX ERZEUGEN - falls gewuenscht muessen beide direkt so eingebunden werden
%\usepackage{index}
%\makeindex


% 	ERZEUGEN DES ABKUERZUNGSVERZEICHNISSES
%    Funktioniert nicht zusammen mit Glossary/Glossaries - aber mit nomenclature
%\usepackage[%
%%			footnote, 				% die Langform als Fußnote ausgeben
%    		nohyperlinks, 			% wenn hyperref geladen ist, wird die Verlinkung unterbunden
%    		printonlyused, 			% nur Abkürzungen auflisten, die tatsächlich verwendet werden.
%%        	withpage, 				% Abkürzungsverzeichnis mit Seitenzahl auf welcher die Abkürzung als 
%        							% erstes verwendet wurde
%%    		smaller, 				% Text soll kleiner erscheinen, das Paket relsize wird vorausgesetzt
%%    		dua, 					% es wird immer die Langform ausgegeben
%%    		nolist 					% es wird keine Liste mit allen Abkürzungen ausgegeben
%			]{acronym}				% Abkürzungsverzeichnis

%	Symbolverzeichnis erzeugen - lade dazu mit %%%%%%%%%%%%%%%%%%%%%%%%%%%%%%%%%%%%%%%%%%%%%%%%%%%%%%%%%%%%%%%%%%%%%%%%%%%%%%%
%  Author     : Dennis Schuethe        Date    : 28.12.2011                   %
%  Filename   : symbolpage.tex         Version : 1.0                          %
%  Information: This file will create the nomenclature and should be designed %
%               the way you want to. There are two options:                   %
%               1. Use the normal nomenclature which is given and created by  %
%                  the nomencl.sty -> see package declarations                %
%               2. To have more than one column for the nomenclature, then    %
%                  use the \begin{multicols}, but this cause a bad look,      %
%                  because the name is printed within the columns, better use %
%                  option 3                                                   %
%               3. Find the nomencl.sty and then rewrite the lines 160 and    %
%                  164 as it is shown below, then you need to add the         %
%                  \chapter*{\nomname} or \section*{\nomname} by yourself     %
%-----------------------------------------------------------------------------%
%  Changelog  :                                                               %
%  28.12.2011  opened this file                                               %
%              may the nomencl.sty could be changed by adding this as option, %
%              need to try a workaround                                       %
%  %
%%%%%%%%%%%%%%%%%%%%%%%%%%%%%%%%%%%%%%%%%%%%%%%%%%%%%%%%%%%%%%%%%%%%%%%%%%%%%%%


%157 \def\thenomenclature{%
%158  \@ifundefined{chapter}%
%159  {
%160    %\section*{\nomname}
%161    \if@intoc\addcontentsline{toc}{section}{\nomname}\fi%
%162  }%
%163  {
%164    %\chapter*{\nomname}
%165    \if@intoc\addcontentsline{toc}{chapter}{\nomname}\fi%
%166  }%

%\renewcommand{\nomname}{newname}	% New name for the nomenclature
%\chapter*{\nomname}				% Do the heading for this chapter(section)
% else			
% newpage							% only if chapter command is not used
%\begin{multicols}{2}				% Do multiple columns - use package multicol
\printnomenclature %[2cm] 			% Symbolverzeichnis mit breite xcm für symbole, std is 1cm
%\end{multicols} das Verzeichnis 
%\usepackage[%
%			intoc,					% Eintrag des Abkuerzungsverzeichnis
%			notintoc,				% Kein Eintrag - default
%			english,				% Ueberschrift in English (default)
%			german,					% in deutsch
%			]{nomencl}
%\makenomenclature					% damit Symbolverzeichnis erzeugt wird
%\usepackage{multicol}				% Zum drucken mehrerer Spalten in einem Environment

% Glossary
%\usepackage[%
%			style    = list,		% list, altlist, super, 	long (default)
%			header   = plain,		% Ausgabe des Kopfes, 		default - none
%			border   = plain,		% Ausgabe einer Umrandung, 	default - none
%			cols     = 3,			% 3 Spalten, Name, Beschreibung, Seite; default = 2
%			number   = none,		% Keine Rückverweise, default = page, angabe anderer counter möglich
%			toc		 = true,		% Fügt Eintrag ins Inhaltsverzeichnis hinzu, defualt false
%			hypertoc = true,		% Setzt Anker direkt vor die Überschrift, toc danach. 
									% Beides gleichzeitig nicht erlaubt. Selbe Funktion. default - false
%			hyper    = true,		% Hyperlink für Seitenzahlen, bzw. bei Acronymen zum Glossary
%			section  = true,		% Section für Überschrift auch wenn chapter defined ist. false default
%			acronym	 = true,		% Seperate Liste für Acronyme zum Glossary, default - false
%			global   = true,		% Acronym-Befehle haben globalen Effekt. defualt - false
%			]{glossary}
% \makeglossary
% Glossaries - same as Glossary but with more options/better handling and individualization
\usepackage[%
%			toc,      				% Fügt Eintrag ins Inhaltsverzeichnis hinzu, defualt false
			nonumberlist,			% keine Ausgabe der Seitenzahlen
%			nowarn,					% unterdrückt alle warungen des pakets
%			nomain,					% kein hauptglossary wird erzeugt, acronym oder newglossary um Einträge zu erzeugen
%			sanitize={%				% unterdrückt die keys für einträge oder fügt sie hinzu
%					  name        = false,%
%					  description = false,%
%					  symbol      = false},%
%			translate = true,		% zur übersetzung von einträgen - siehe doku, default false
			hyperfirst = false,		% erzeugt hyperlink zum glossar bei erster verwendung, default true
%			hyper = false,			% Hyperlinks ausschalten
%			numberline,				% Eintrag ins toc mit Nummerierung
%			section=section,		% in welcher ebene der eintrag, default chapter if loaded sonst section
%			style = long,			% default list, sonst: long, super, tree
%			nolong,					% pakete für longtable werden nicht geladen und befehle nicht definiert
%			nosuper,				% supertabular wird nicht geladen, pakete nicht definiert
%			nolist,					% list wird nicht geladen definiert
%			notree,					% gloassrytree wird nicht geladen definiert
%			nostyles,				% keine styles werden geladen, eigene definieren
%			counter = page,			% page default, or any other counter
%			sort = def,				% def - sortiert nach definition, standard(default)-sortiert nach sort key, sonst name, use - sortieren nach reihenfolge wie sie im dokument vorkommen
			%%%%%%%%%%%%% Acronym options
			acronym,  				% Seperate Liste für Acronyme zum Glossary, default - false
%			acronymlist={},			% wenn mehr als eine abkürzungsliste, muss diese hier mit angegeben werden damit das glossary als acronym behandelt wird
%			description,			% erlaubt eine zusätzliche beschreibung
%			footnote,				% schreibt die long version als fußnote bei erster nutzung
%			smallcaps,				% acronyme werden als kapitelchen angezeigt
%			smaller,				% kleinere schrift, laden von relsize für befehl textsmaller erforderlich
%			dua,					% langform wird immer ausgegeben
			shortcuts,				% um selbe befehle wie acronym paket zu haben
			]{glossaries}
%\usepackage{relsize}				% in verbindung mit smaller - stellt textsmaller befhel zur verfügung
%Ein eigenes Symbolverzeichnis erstellen
\newglossary[slg]{symbols}{syi}{syg}{Symbolverzeichnis}
\makeglossaries
%%%%%%%%%%%%%%%%%%%%%%%%%%%%%%%%%%%%%%%
% Loading label options               %
%%%%%%%%%%%%%%%%%%%%%%%%%%%%%%%%%%%%%%%
\usepackage[ngerman]
			{varioref}	            % To get automatic "see Section xy on page xx"
\usepackage[german] %
			{fancyref}	    		% same as varioref, except this is setting section automatically
									% Options in []-brackets:
									% language, e.g. english, ngerman - last entry is main-language
									% paren - sets the reference in ()-brackets
									% plain - no pagenumber in the reference
									% -----------------------------------------
									% standard prefixes used by fancyref:
									% sec, chap, part, fig, tab, enum, eq, fn
\fancyrefchangeprefix{\fancyrefchaplabelprefix}{cha} % sets a new prefix for the label chap

%%%%%%%%%%%%%%%%%%%%%%%%%%%%%%%%%%%%%%%
% Loading unit options                %
%%%%%%%%%%%%%%%%%%%%%%%%%%%%%%%%%%%%%%%
\usepackage[%				
%   group-separator = {.}, 			 % group-seperator={,} -> 1,345,234.23
%   round-mode 		= places,		 % round-mode= places(2digits), figures(1digit)
%   round-precision = 3,			 % round-precision= x , xdigits are rounded to
	binary-units 	= true,			 % Laden von \byte \bit, \kibi usw. (false default)
	locale = DE,					 % lacale= DE, uses german
%	loctolang={UK:english, DE:ngerman},	% loctolang={USA:USenglish,DE:ngerman}, if language changed in
									 % document
%			detect-all,				 % detect-all : uses the text options not the math mode to set
									 % the letters/numbers
			]{siunitx}				 % For typical Units in Text- and Mathmode

%%%%%%%%%%%%%%%%%%%%%%%%%%%%%%%%%%%%%%%
% Loading sourcecode options          %
%%%%%%%%%%%%%%%%%%%%%%%%%%%%%%%%%%%%%%%
\usepackage[%
			hyperref 	= false,	% switch off hyperref hack
			float 		= false,	% switch off float hack
			listings 	= true,  	% switch off listings hack
			]{scrhack}				% to get rid off the warning @addtocbasic
\usepackage{listings}				% options : basicstyle= e.g. \ttfamily
									% fontadjust=true % before load package [scaled=0.78]{luximono}
									% columns=flexible




%\hyphenation{}						 % Trennung von Woertern falls nicht automatisch korrekt erkannt
%%%%%%%%%%%%%%%%%%%%%%%%%%%%%%%%%%%%%%%
% Loading table options		          %
%%%%%%%%%%%%%%%%%%%%%%%%%%%%%%%%%%%%%%%
\usepackage{multirow}				% Fuer Mehrzeilige Zellen
%\usepackage{array}					% Festlegen von Breiten, Präfixe, Suffixe
% see also package siunitx - damit Spalten mit Zahlen auf bestimmte Länge ausgerichtet werden
\usepackage{tabularx}				% Zum festlegen der Gesamtbreite der Tabelle & verwenden von X fuer 
									% variable Spaltenbreite
%\usepackage{longtable}				% Tabelle ueber mehr als eine Seite
\usepackage{ltxtable}				% longtable + tabularx eigenschaften
\usepackage{booktabs}				% Fuer besseres aussehen der Linien
%\usepackage{rotating}				% Querformat der Tabelle
									% !!! Vorsicht !!!
									% tablecaptionabove wird nicht ausgewertet,
									% Einfuegen von \vskip\abovecaptionskip --> Nur unter KOMA Script
\usepackage{ctable}					% For tables with footnote under the table directly
%%%%%%%%%%%%%%%%%%%%%%%%%%%%%%%%%%%%%%%
% Hyperref for PDFLATEX with links    %
%%%%%%%%%%%%%%%%%%%%%%%%%%%%%%%%%%%%%%%
\usepackage[draft 		=	false,	% all hypertext options are turned off
			final 		=	true, 	% all hypertext options are turned on
			raiselinks 	= 	true,	% Allow links to reflect real height - e.g. with pictures
			breaklinks	=	false,	% false=Do not break a line within a link
			backref		=	false,	% false=no backlinks in the bibliography
			pagebackref =	false,	% false=no pagebacklinks in bibliography
			linktocpage =	false,	% true=makes page-no linked and not text in TOC,LOF and LOT
			colorlinks	=	true,	% true=Colors the text of links and anchors
%			linkcolor 	=	red, 	% Color for normal internal links.
%			anchorcolor =	black, 	% Color for anchor text.
%			citecolor 	=	green, 	% Color for bibliographical citations in text.
%			filecolor 	=	cyan, 	% Color for URLs which open local files.
%			menucolor 	=	red,	% Color for Acrobat menu items.
%			runcolor 	= filecolor,% Color for run links (launch annotations).
			allcolors 	= 	red,	% Set all color options (without border and field options).
			urlcolor 	=	blue,	% Color for linked URLs.
			%allbordercolors  =red,	% Set all colors
%			citebordercolor =green,	% cite color
%			filebordercolor =cyan,	% file color
%			linkbordercolor =red,	% link color
			%urlbordercolor  =blue,	% url color
%			pdfborder = 0 0 0,		% weder farbige links noch umrandung			
%			menubordercolor =red,	% set menu color
%			frenchlinks = 	false,	% Use small caps instead of color for links.
			bookmarks	=	true,	% true=Bookmarks are added to the pdf
			bookmarksopen=	false,	% false=Bookmarks not open when opening the pdf
			bookmarksnumbered=true,	% true= Eintraege sind nummeriert
%			pdfpagemode	=	FullScreen,	% File is opened in Full Screen
%			pdfstartview=	fit,	% Fit size when opening pdf
			pdfpagelabels=	true,	% true = Roemische Zahlen usw werden dargestellt,
									% false= fortlaufende nummerierung
			]{hyperref}

%%%%%%%%%%%%%%%%%%%%%%%%%%%%%%%%%%%%%%%%%%
%	Usepackage für Pseudocode 			 %
%%%%%%%%%%%%%%%%%%%%%%%%%%%%%%%%%%%%%%%%%%

%\usepackage{algorithm}
\usepackage[Algorithmus]{algorithm} 	% deutsche notation im titel
\usepackage[noend]{algpseudocode}