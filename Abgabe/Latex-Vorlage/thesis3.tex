%---------------------------------------------------------------------------------------------------
% Voreinstellungen (Layout, neue Befehle, etc.)
%---------------------------------------------------------------------------------------------------

%---------------------------------------------------------------------------------------------------
% Einstellungen
% (gelten nur in Zusammenarbeit mit pdflatex)
%---------------------------------------------------------------------------------------------------
\documentclass[
  pagesize,	                                           % flexible Auswahl des Papierformats
  a4paper,  	                                         % DIN A4
  oneside,    	                                       % einseitiger Druck
  BCOR5mm,      	                                     % Bindungskorrektur
  headsepline,                                         % Strich unter der Kopfzeile
  12pt,                                                % 12pt Schriftgr��e
	halfparskip,                                         % Europ�ischer Satz: Abstand zwischen Abs�tzen
	abstracton,																					 % Spezielle Formatierung, die erlaubt, dass die
																											 % Zusammenfassung vor dem Inhaltsverzeichnis steht
	%draft,																							 % Es handelt sich um eine Vorabversion	
	final,																							 % Es handelt sich um die endg�ltige Version
	liststotoc,																					 % Tabellen- und Abbildungsverzeichnis im 																																 % Inhaltsverzeichnis
	idxtotoc,																						 % Index im Inhaltsverzeichnis	
  bibtotoc,                                            % Literaturverzeichnis im Inhaltsverzeichnis  
]{scrreprt}                                            % KOMA-Scriptklasse Report

%---------------------------------------------------------------------------------------------------
\usepackage[english,ngerman]{babel}                    % deutsche Trennmuster
\usepackage[T1]{fontenc}                               % EC-Schriften, Trennstellen nach Umlauten
\usepackage[latin1]{inputenc}                          % direkte Umlauteingabe (� statt "a)
                                                       % latin1/latin9 f�r unixoide Systeme
                                                       % (latin1 ist auch unter Win verwendbar)
                                                       % ansinew f�r Windows
                                                       % applemac Macs
                                                       % cp850 OS/2
\usepackage{times}              			 % Schriften Paket
\usepackage{array,ragged2e} 				 % Wichtig f�r Abstandsformatierung

%---------------------------------------------------------------------------------------------------
\usepackage{cmbright}                                  % serifenlose Schrift als Standard
                                                       % + alle f�r TeX ben�tigten mathematischen
                                                       %   Schriften einschlie�lich der AMS-Symbole
\usepackage[scaled=.90]{helvet}                        % skalierte Helvetica als \sfdefault
\usepackage{courier}                                   % Courier als \ttdefault

%---------------------------------------------------------------------------------------------------
\usepackage[automark]{scrpage2}                        % Anpassung der Kopf- und Fu�zeilen
\usepackage{xspace}                                    % Korrekter Leerraum nach Befehlsdefinitionen
\usepackage{setspace}				  % Dieses Package brauchen wir f�r den 				
\usepackage[pdftex]{graphicx}
\usepackage[absolute,overlay]{textpos}         
\usepackage[final]{pdfpages}																											 % anderthalbzeiligen Abstand.
\usepackage{natbib}                                    % Neuimplementierung des \cite-Kommandos
\usepackage{bibgerm}       											       % Deutsche Bezeichnungen
\usepackage[absolute]{textpos}                         % placing boxes at absolute positions
\usepackage[final]{pdfpages}                           % include pages of external PDF documents
\usepackage{tabularx}                                  % Spaltenbreite bis zur Seitenbreite dehnen
\usepackage{makeidx}													% Paket zur Erstellung eines Stichwortverzeichnisses
\makeindex																						 % Automatische Erstellung des Stichwortverzeichnis
\usepackage[intoc,
						german,
						prefix]{nomencl}
\makenomenclature

%---------------------------------------------------------------------------------------------------
 \usepackage{graphicx}                                 % Zur Einbindung von PDF-Bildern
 \usepackage[colorlinks,			 % Einstellen und Laden des Hyperref-Pakets
	pdftex,
	bookmarks,
	bookmarksopen=false,
	bookmarksnumbered,
	citecolor=blue,
	linkcolor=blue,
	urlcolor=blue,
	filecolor=blue,
	linktocpage,
  pdfstartview=Fit,                                  % startet mit Ganzseitenanzeige    
	pdfsubject={Multiagentensystemgest�tzte Clusteranalyse},
	pdftitle={Diplomarbeit im Fachbereich Elektrotechnik \& Informatik an der HAW-Hamburg},
	pdfauthor={Bj�rn Jensen, http://www.mirou.de}]{hyperref}
 \pdfcompresslevel=9
 
%---------------------------------------------------------------------------------------------------
% Inhaltsverzeichnis und Abschnittnummerierung
%---------------------------------------------------------------------------------------------------
\setcounter{secnumdepth}{2}   % Ich habe recht kurze Kapitel. Die sollen nicht durchnummeriert sein.
\setcounter{tocdepth}{2}

%---------------------------------------------------------------------------------------------------
% Abbildungsverzeichnis
%---------------------------------------------------------------------------------------------------
\graphicspath{{graphics/}}

%---------------------------------------------------------------------------------------------------
% Kopf- und Fu�zeilen
%---------------------------------------------------------------------------------------------------
\pagestyle{scrheadings}
\clearscrheadings
\clearscrplain
\clearscrheadfoot
\ohead{\pagemark}
\ihead{\headmark}

%---------------------------------------------------------------------------------------------------
% Neue Befehle
%---------------------------------------------------------------------------------------------------
\input{einstellungen/neuebefehle.sty}

%---------------------------------------------------------------------------------------------------
% Trennung
%---------------------------------------------------------------------------------------------------
\input{einstellungen/trennungen.sty}

%---------------------------------------------------------------------------------------------------
% Anpassung der Parameter, die TeX bei der Berechnung der Zeilenumbr�che verwendet:
%---------------------------------------------------------------------------------------------------
\tolerance 1414
\hbadness 1414
\emergencystretch 1.5em
\hfuzz 0.3pt
\widowpenalty=10000
\vfuzz \hfuzz
\raggedbottom															% Die stilistischen Parameter

%--------------------------------------------------------------------------------------------------- 
% Anfang des Schriftst�cks
%---------------------------------------------------------------------------------------------------	
\begin{document}

%--------------------------------------------------------------------------------------------------- 
% Erstellen des Deck- und des Titelblatts
%---------------------------------------------------------------------------------------------------
	\createCoverAndTitlePage{Bachelor}																			% Art der Arbeit
													{thesis}																			% Bezeichnung arbeit oder thesis 
													{Martin Mustermann}														% Author
													{Entwicklung und Aufbau eines 
													 mikrorechnergesteuerten 
													 Best�ckungsautomaten}											  % Titel				
													{Informations- und Elektrotechnik}						% Studiengang
													{Prof.\ Dr.\ rer.\ nat.\ Martin Zapf}					% Erstgutachter
													{Prof.\ Dr.\-Ing.\ Armin Kluge}								% Zweitgutachter
													

  \createAbstract					{Bachelor}																			% Art der Arbeit
													{thesis}																			% Bezeichnung arbeit oder thesis 
													{Martin Mustermann}														% Author
													{Entwicklung und Aufbau eines 
													 mikrorechnergesteuerten 
													 Best�ckungsautomaten}											  % Titel				
													{Development and Construction of a Microprocessor 
													controlled allocation processor}							% Titel Englisch
													{Steuerung, und viele weitere interessante 
													Stichwort}																		% Stichworte
													{Controller, Microprocessor, and other 
													interesting words describing the whole 
													process}																			% Keywords (Stichworte Emglisch)
													{Diese Arbeit umfasst alles was man mit einem 
													Mikrorechner machen kann und nat�rlich noch vieles mehr.
													etc.}																					% Kurzzusammenfassung
													{Inside this report the construction of a very 
													important Controller for microproc-essors is 
													described.												
													etc.}																			% Abstract (Kurzzusammenfassung Englisch)

												
%--------------------------------------------------------------------------------------------------- 
% Zusammenfassung
%---------------------------------------------------------------------------------------------------			
  									  													


%--------------------------------------------------------------------------------------------------- 
% Danksagung  
%---------------------------------------------------------------------------------------------------	
	\input{standard/danke}  																							

%--------------------------------------------------------------------------------------------------- 
% Verzeichnisse
%---------------------------------------------------------------------------------------------------	
  \tableofcontents                              												% Inhaltsverzeichnis
	\listoftables                                 												% Tabellenverzeichnis
	\listoffigures                                												% Abbildungsverzeichnis  
	
%--------------------------------------------------------------------------------------------------- 
% Der erste Teil der Arbeit:
%---------------------------------------------------------------------------------------------------
	%---------------------------------------------------------------------------------------------------
% Einf�hrung
%---------------------------------------------------------------------------------------------------
\newpage
%\part{Anfang}
\chapter{Einf�hrung}
Hier muss was zur \addIndexEntry{Einf�hrung} erz�hlt werden.
	
%---------------------------------------------------------------------------------------------------	
% Der zweite Teil der Arbeit:
%---------------------------------------------------------------------------------------------------
	%---------------------------------------------------------------------------------------------------
% Hauptteil
%---------------------------------------------------------------------------------------------------
\newpage
%\part{Hauptteil}
%---------------------------------------------------------------------------------------------------
% Analyse
%---------------------------------------------------------------------------------------------------
\newpage
\chapter{Analyse}
Hier wird dann analysiert
						% Kapitel 2: Analyse
%---------------------------------------------------------------------------------------------------
% Design
%---------------------------------------------------------------------------------------------------
\newpage
\chapter{Design}
Hier wird designed\nomenclature{Design}{Und eine Erkl�rung}!

\begin{figure}[htbp]
	\centering	\includegraphics{grafiken/listen-tiny.jpg}
	\caption{GNU.org}
	\label{fig:Gnu}
\end{figure}
							% Kapitel 3: Design
%---------------------------------------------------------------------------------------------------
% Realisierung
%---------------------------------------------------------------------------------------------------
\newpage
\chapter{Realisierung}
Und hier wird realisiert...was, das bleibt jedem selbst �berlassen...
	% Kapitel 4: Realisierung

%---------------------------------------------------------------------------------------------------
% Der dritte Teil der Arbeit
%---------------------------------------------------------------------------------------------------
%	%---------------------------------------------------------------------------------------------------
% Schluss
%---------------------------------------------------------------------------------------------------
\newpage
%\part{Schluss}
\chapter{Schluss}
Ber�hmte letzte Worte...
																								
%---------------------------------------------------------------------------------------------------	
% Literaturverzeichnis
%---------------------------------------------------------------------------------------------------		
	\bibliographystyle{dinat}        		    														% Anpassung an deutsche Zitierweise
                                          														% Alphabetische Sortierung, Abk�rzungen
  \bibliography{literatur/literatur}      														% Literaturverzeichnis
  % Zitate die in den Referenzen aufscheinen, aber nicht unbedingt im Text zitiert werden m�ssen
% Literatur, die mit \nocite oder \cite zitiert wird, mu� im .bib File erfasst werden!
\nocite{Guenther:2002}
\nocite{Lamport:1995}
\nocite{Goossens:2000}
\nocite{Kohm:2003}
\nocite{Hunt:2003}
\nocite{Boehm:2002}
\nocite{Schmatz:2004}
\nocite{Streitz:2005}
\nocite{HP:2004}
\nocite{Luede:2004}
\nocite{Demarco:1999}
\nocite{Kollakowski:2004}
\nocite{Dawson:2003}
\nocite{Poenicke:1988}
\nocite{Kruse:2000}
\nocite{Nilsson:1998}
\nocite{Heinsohn:1999}
\nocite{Luger:2001}
\nocite{Kuehnel:2001}
\nocite{Bigus:2001}
\nocite{Ferber:2001}
\nocite{Wooldridge:2002}
\nocite{KuroseRoss:2002}
\nocite{Vogt:2001}
\nocite{Roetzer:1999}
\nocite{P3P:2004}
\nocite{CPEX:2004}
\nocite{Duden:1997}
\nocite{Hoerauf:2001}
\nocite{Schirru:2004}
\nocite{Babic:2003}
\nocite{Luepke:2004}																				% hier k�nnen alle Schriftst�cke aufgef�hrt werden, die nicht zitiert, aber dennoch nennenswert sind!

%---------------------------------------------------------------------------------------------------	
% Anh�nge
%---------------------------------------------------------------------------------------------------	
	\appendix
% \input{anhang/hilfsmittel/hilfsmittel}															% Anhang A: Hilfsmittel zur Erstellung
																																			% 					dieser Arbeit
%	\input{anhang/quellcode/quellcode}																	% Anhang B: Quellcode

%---------------------------------------------------------------------------------------------------	
% Glossar
%---------------------------------------------------------------------------------------------------	
	\printnomenclature

%---------------------------------------------------------------------------------------------------	
% Stichwortverzeichnis
%---------------------------------------------------------------------------------------------------	
	\printindex
	
%---------------------------------------------------------------------------------------------------	
% Erkl�rung �ber Selbstst�ndigkeit
%---------------------------------------------------------------------------------------------------		
	\asurency	

%--------------------------------------------------------------------------------------------------- 
% Ende des Schriftst�cks
%--------------------------------------------------------------------------------------------------- 
\end{document}
