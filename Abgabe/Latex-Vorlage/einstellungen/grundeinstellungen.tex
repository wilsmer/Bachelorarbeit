%---------------------------------------------------------------------------------------------------
% Einstellungen
% (gelten nur in Zusammenarbeit mit pdflatex)
%---------------------------------------------------------------------------------------------------
\documentclass[
  pagesize,	                                           % flexible Auswahl des Papierformats
  a4paper,  	                                         % DIN A4
  oneside,    	                                       % einseitiger Druck
  BCOR5mm,      	                                     % Bindungskorrektur
  headsepline,                                         % Strich unter der Kopfzeile
  12pt,                                                % 12pt Schriftgr��e
	halfparskip,                                         % Europ�ischer Satz: Abstand zwischen Abs�tzen
	abstracton,																					 % Spezielle Formatierung, die erlaubt, dass die
																											 % Zusammenfassung vor dem Inhaltsverzeichnis steht
	%draft,																							 % Es handelt sich um eine Vorabversion	
	final,																							 % Es handelt sich um die endg�ltige Version
	liststotoc,																					 % Tabellen- und Abbildungsverzeichnis im 																																 % Inhaltsverzeichnis
	idxtotoc,																						 % Index im Inhaltsverzeichnis	
  bibtotoc,                                            % Literaturverzeichnis im Inhaltsverzeichnis  
]{scrreprt}                                            % KOMA-Scriptklasse Report

%---------------------------------------------------------------------------------------------------
\usepackage[english,ngerman]{babel}                    % deutsche Trennmuster
\usepackage[T1]{fontenc}                               % EC-Schriften, Trennstellen nach Umlauten
\usepackage[latin1]{inputenc}                          % direkte Umlauteingabe (� statt "a)
                                                       % latin1/latin9 f�r unixoide Systeme
                                                       % (latin1 ist auch unter Win verwendbar)
                                                       % ansinew f�r Windows
                                                       % applemac Macs
                                                       % cp850 OS/2
\usepackage{times}              			 % Schriften Paket
\usepackage{array,ragged2e} 				 % Wichtig f�r Abstandsformatierung

%---------------------------------------------------------------------------------------------------
\usepackage{cmbright}                                  % serifenlose Schrift als Standard
                                                       % + alle f�r TeX ben�tigten mathematischen
                                                       %   Schriften einschlie�lich der AMS-Symbole
\usepackage[scaled=.90]{helvet}                        % skalierte Helvetica als \sfdefault
\usepackage{courier}                                   % Courier als \ttdefault

%---------------------------------------------------------------------------------------------------
\usepackage[automark]{scrpage2}                        % Anpassung der Kopf- und Fu�zeilen
\usepackage{xspace}                                    % Korrekter Leerraum nach Befehlsdefinitionen
\usepackage{setspace}				  % Dieses Package brauchen wir f�r den 				
\usepackage[pdftex]{graphicx}
\usepackage[absolute,overlay]{textpos}         
\usepackage[final]{pdfpages}																											 % anderthalbzeiligen Abstand.
\usepackage{natbib}                                    % Neuimplementierung des \cite-Kommandos
\usepackage{bibgerm}       											       % Deutsche Bezeichnungen
\usepackage[absolute]{textpos}                         % placing boxes at absolute positions
\usepackage[final]{pdfpages}                           % include pages of external PDF documents
\usepackage{tabularx}                                  % Spaltenbreite bis zur Seitenbreite dehnen
\usepackage{makeidx}													% Paket zur Erstellung eines Stichwortverzeichnisses
\makeindex																						 % Automatische Erstellung des Stichwortverzeichnis
\usepackage[intoc,
						german,
						prefix]{nomencl}
\makenomenclature

%---------------------------------------------------------------------------------------------------
 \usepackage{graphicx}                                 % Zur Einbindung von PDF-Bildern
 \usepackage[colorlinks,			 % Einstellen und Laden des Hyperref-Pakets
	pdftex,
	bookmarks,
	bookmarksopen=false,
	bookmarksnumbered,
	citecolor=blue,
	linkcolor=blue,
	urlcolor=blue,
	filecolor=blue,
	linktocpage,
  pdfstartview=Fit,                                  % startet mit Ganzseitenanzeige    
	pdfsubject={Multiagentensystemgest�tzte Clusteranalyse},
	pdftitle={Diplomarbeit im Fachbereich Elektrotechnik \& Informatik an der HAW-Hamburg},
	pdfauthor={Bj�rn Jensen, http://www.mirou.de}]{hyperref}
 \pdfcompresslevel=9
 
%---------------------------------------------------------------------------------------------------
% Inhaltsverzeichnis und Abschnittnummerierung
%---------------------------------------------------------------------------------------------------
\setcounter{secnumdepth}{2}   % Ich habe recht kurze Kapitel. Die sollen nicht durchnummeriert sein.
\setcounter{tocdepth}{2}

%---------------------------------------------------------------------------------------------------
% Abbildungsverzeichnis
%---------------------------------------------------------------------------------------------------
\graphicspath{{graphics/}}

%---------------------------------------------------------------------------------------------------
% Kopf- und Fu�zeilen
%---------------------------------------------------------------------------------------------------
\pagestyle{scrheadings}
\clearscrheadings
\clearscrplain
\clearscrheadfoot
\ohead{\pagemark}
\ihead{\headmark}

%---------------------------------------------------------------------------------------------------
% Neue Befehle
%---------------------------------------------------------------------------------------------------
\input{einstellungen/neuebefehle.sty}

%---------------------------------------------------------------------------------------------------
% Trennung
%---------------------------------------------------------------------------------------------------
\input{einstellungen/trennungen.sty}

%---------------------------------------------------------------------------------------------------
% Anpassung der Parameter, die TeX bei der Berechnung der Zeilenumbr�che verwendet:
%---------------------------------------------------------------------------------------------------
\tolerance 1414
\hbadness 1414
\emergencystretch 1.5em
\hfuzz 0.3pt
\widowpenalty=10000
\vfuzz \hfuzz
\raggedbottom