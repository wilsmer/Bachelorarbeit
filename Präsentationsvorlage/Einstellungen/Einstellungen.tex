\pdfminorversion=4
\documentclass[9pt]{beamer} 
\setbeamertemplate{navigation symbols}{}
%\setbeamertemplate{frametitle}{}

%\usetheme{Montpellier}
%\setbeamertemplate{footline}[frame number]

\usepackage{pgfpages}
\usepackage{booktabs}
\usepackage[%				
%   group-separator = {.}, 			 % group-seperator={,} -> 1,345,234.23
%   round-mode 		= places,		 % round-mode= places(2digits), figures(1digit)
%   round-precision = 3,			 % round-precision= x , xdigits are rounded to
	binary-units 	= true,			 % Laden von \byte \bit, \kibi usw. (false default)
	locale = DE,					 % lacale= DE, uses german
	per=slash,
%	loctolang={UK:english, DE:ngerman},	% loctolang={USA:USenglish,DE:ngerman}, if language changed in
									 % document
%			detect-all,				 % detect-all : uses the text options not the math mode to set
									 % the letters/numbers
			]{siunitx}				 % For typical Units in Text- and Mathmode
%%%%%%%%%%%%%%%%%%%%%%%%%%%%%%%%%%%%%%%
% Loading mathematical writing style  %
%%%%%%%%%%%%%%%%%%%%%%%%%%%%%%%%%%%%%%%
\usepackage{mathtools}
    
\mode<presentation>
{
%\usetheme{Madrid}
\usetheme{Boadilla}

\setbeamercovered{transparent}
}
\usepackage{subfigure} 
\usepackage{amsmath}  			% erleichtert Mathe 
\usepackage{enumerate}			% schicke Nummerierung
\usepackage{graphicx} 			% für Grafik-Einbindung
\usepackage{multicol}
\usepackage{lmodern}
\usepackage[ngerman]{babel}
\usepackage[utf8]{inputenc}

\usepackage{listings}
\lstset{ %
  backgroundcolor=\color{white},   % choose the background color; you must add \usepackage{color} or \usepackage{xcolor}; should come as last argument
  basicstyle=\footnotesize,        % the size of the fonts that are used for the code
  breakatwhitespace=false,         % sets if automatic breaks should only happen at whitespace
  breaklines=true,                 % sets automatic line breaking
  captionpos=b,                    % sets the caption-position to bottom
  commentstyle=\color{mygreen},    % comment style
  deletekeywords={...},            % if you want to delete keywords from the given language
  escapeinside={\%*}{*)},          % if you want to add LaTeX within your code
  extendedchars=true,              % lets you use non-ASCII characters; for 8-bits encodings only, does not work with UTF-8
  frame=single,	                   % adds a frame around the code
  keepspaces=true,                 % keeps spaces in text, useful for keeping indentation of code (possibly needs columns=flexible)
  keywordstyle=\color{blue},       % keyword style
  language=Matlab,                 % the language of the code
  morekeywords={*,...},            % if you want to add more keywords to the set
  numbers=left,                    % where to put the line-numbers; possible values are (none, left, right)
  numbersep=5pt,                   % how far the line-numbers are from the code
  numberstyle=\tiny\color{mygray}, % the style that is used for the line-numbers
  rulecolor=\color{black},         % if not set, the frame-color may be changed on line-breaks within not-black text (e.g. comments (green here))
  showspaces=false,                % show spaces everywhere adding particular underscores; it overrides 'showstringspaces'
  showstringspaces=false,          % underline spaces within strings only
  showtabs=false,                  % show tabs within strings adding particular underscores
  stepnumber=2,                    % the step between two line-numbers. If it's 1, each line will be numbered
  stringstyle=\color{mymauve},     % string literal style
  tabsize=2,	                   % sets default tabsize to 2 spaces
  title=\lstname                   % show the filename of files included with \lstinputlisting; also try caption instead of title
}

\usepackage{media9}
%\usepackage{multimedia}

%=================== Set Beamer Templates ====================================
%\definecolor{DOGGYbg}{RGB}{167,183,208} %HAW Color 
%\definecolor{DOGGY}{RGB}{26,60,116} % HAW Color 2
%\definecolor{DOGGYbg}{RGB}{240,240,255} %HAW Color 
%\definecolor{DOGGY}{RGB}{26,60,116} % HAW Color 2
%\definecolor{dgreen}{RGB}{0,150,0} % HAW Color 2
%\definecolor{lightgray}{RGB}{150,150,150} % HAW Color 2

%\setbeamercolor*{structure}{fg=DOGGY, bg=DOGGYbg}
%\setbeamercolor*{palette primary}{use=structure,fg=structure.bg,bg=structure.fg!95!black}
%\setbeamercolor*{palette secondary}{use=structure,fg=structure.bg,bg=structure.fg!70!black}
%\setbeamercolor*{palette tertiary}{use=structure,fg=structure.bg,bg=structure.fg!95!black}
%\setbeamercolor*{palette quaternary}{fg=structure.bg,bg=black}
%\setbeamercolor*{block title}{use=structure,fg=structure.bg,bg=structure.fg!75!black}

\definecolor{DOGGYbg}{RGB}{80,90,100} %HAW Color
\definecolor{DOGGY}{RGB}{26,60,116} % HAW Color 2
\definecolor{lgray}{RGB}{240,240,240} %HAW Color
\definecolor{darkgreen}{RGB}{50,150,50} %HAW Color

\definecolor{mygreen}{rgb}{0,0.6,0}
\definecolor{mygray}{rgb}{0.5,0.5,0.5}
\definecolor{mymauve}{rgb}{0.58,0,0.82}

\setbeamercolor*{structure}{fg=DOGGY, bg=DOGGYbg} 
\setbeamercolor*{palette primary}{use=structure,fg=structure.fg,bg=lgray}
\setbeamercolor*{palette secondary}{use=structure,fg=structure.fg,bg=lgray!90!black}
\setbeamercolor*{palette tertiary}{use=structure,fg=structure.fg,bg=lgray}
\setbeamercolor*{palette quaternary}{fg=structure.fg,bg=lgray}
\setbeamercolor*{block title}{use=structure,fg=structure.fg,bg=lgray}


\setbeamertemplate{navigation symbols}{} % No navigation symbols
\setbeamercolor{navigation symbols}{use=structure, fg=structure.bg,
bg=structure.bg}
% New footline
\defbeamertemplate{footline}{rownav}{%
  \leavevmode%
  \hbox{%
  \begin{beamercolorbox}[wd=.15\paperwidth,ht=2.5ex,dp=1ex,center]{author in
  head/foot}% only put the author and dont use the institute name - width 1/4
    \usebeamerfont{author in head/foot}\insertshortauthor
  \end{beamercolorbox}%
  \begin{beamercolorbox}[wd=.60\paperwidth,ht=2.5ex,dp=1ex,center]{title in
  head/foot}%
    \usebeamerfont{title in head/foot}\insertshorttitle
  \end{beamercolorbox}%
  \begin{beamercolorbox}[wd=.08\paperwidth,ht=2.5ex,dp=1ex,right]{date in
  head/foot}%
    \usebeamercolor[fg]{navigation symbols}% 
    \insertslidenavigationsymbol\hspace*{1ex}%
    \vspace*{-0.5ex}
  \end{beamercolorbox}%
  \begin{beamercolorbox}[wd=.17\paperwidth,ht=2.5ex,dp=1ex,center]{date in
  head/foot}%
    \usebeamerfont{date in head/foot}\insertshortdate{}\hspace*{2em}
    \textbf{\insertframenumber{}}% 
  \end{beamercolorbox}%
    }%
  \vskip0pt%
}
\setbeamertemplate{footline}[rownav]{}

\defbeamertemplate*{title page}{myPage}[1][]
{
  \vbox{}
  \vfill
  \begin{centering}
    \begin{beamercolorbox}[sep=8pt,center,#1]{author}
      \usebeamerfont{title}\inserttitle
    \end{beamercolorbox}%
    \vskip1em\par
    \begin{beamercolorbox}[sep=8pt,center,#1]{author}
      \usebeamerfont{author}\insertauthor
    \end{beamercolorbox}
    \begin{beamercolorbox}[sep=8pt,center,#1]{institute}
      \usebeamerfont{institute}\insertinstitute
    \end{beamercolorbox}
    \begin{beamercolorbox}[sep=8pt,center,#1]{date}
      \usebeamerfont{date}\insertdate
    \end{beamercolorbox}
  \end{centering}
  \vfill
}
 
\setbeamertemplate{title page}[myPage]{}
% ============== Document begin ==============================================

