%\usepackage[T1]{fontenc}			% Schriften fuer Europaeische Zeichen passend Kodiert
\usepackage[utf8]{inputenc}        		% Eingabe von Umlauten, ss usw. - 
						% utf8: kann nicht jeder Editor speicher, aber plattformuebergreifend
						% latin1: Unix, VMS, Windows
						% ansinew: Windows
						% latin9: wie latin1, jedoch mit Eurozeichen
						
\usepackage[					% Passt Konventionen an Sprache an
	     %english, 
	     ngerman				% ngerman: Neue Deutsche Rechtschreibung
	    ]{babel}				% z.B. UKenglish, USenglish, canadian, german, austrian, naustrian

\geometry{paperwidth=150mm,paperheight=105mm}

% load packages for european, espacially german users

%\usepackage[T1]{fontenc}			% Schriften fuer Europaeische Zeichen passend Kodiert
%\usepackage[utf8]{inputenc}        		% Eingabe von Umlauten, ss usw. - 
						% utf8: kann nicht jeder Editor speicher, aber plattformuebergreifend
						% latin1: Unix, VMS, Windows
						% ansinew: Windows
						% latin9: wie latin1, jedoch mit Eurozeichen
						
%\usepackage[					% Passt Konventionen an Sprache an
%	     english, 
%	     ngerman				% ngerman: Neue Deutsche Rechtschreibung
%	    ]{babel}				% z.B. UKenglish, USenglish, canadian, german, austrian, naustrian
	    
% Schriftfamilien Laden
%\usepackage{lmodern}				% Laedt die Schriftart Latin-Modern fuer Text
%\usepackage{courier}
%\rmfamily
%\sffamily
%\ttfamily
%\renewcommand{\rmdefault}{
%			  pbk	% Bookman
%			  phv	% Helvetica
%			  cmr	% CM Roman
%			  ppl	% Palatino
%			  ptm	% Times Roman
%			  pag	% Avant Garde
%			  pcz	% Zapf Chancery
%			 }
%\renewcommand{\familydefault}{\sfdefault}

% Alternativ Schriftart (Textshchrift) ändern über usepackage:
%\usepackage{
%	    mathpazo	% Palatino
%	    mathptmx	% Times
%	    avant	% Avant Garde
%	    courier	% Courier
%	    chancery	% Zapf Chancery
%	    bookman	% Bookman
%	    newcent	% New Century Schoolbook
%	    charter	% Charter
%	    helvet	% Helvetica
%} 
%\usepackage[scaled=0.92]{helvet}		% Helvetica, ist größer als Times und sollte bei verwendung beider skaliert werden.

% Im Dokument wechslen:
%\fontfamily{pbk}\selectfont

%\usepackage[
%	    babel,
%	    german=guillemets,  		% franz. <<  >>
%	    german= quotes,			% deutsche Anfuehrungszeichen "
%	   ]{csquotes}				% Laden der richtigen Anfuehrungszeicehn fuer deutsche sprache
						% = quotes fuer englische sprache



	   
%\usepackage{setspace}				% Fuer Abstaende im Dokument
%\onehalfspacing				% aendert Zeilenabstand auf 1 1/2
									
%\usepackage{microtype}				% optischer Randausgleich in PDF's
									% ungeeignet fuer internettexte o.ae.	
	   
%\usepackage[					% Papierformate auf denen gedruckt werden soll
%		a0,b0,
%		a1,b1,
%		a2,b2,
%		a3,b3,
%		a4,b4,
%		a5,b5,
%		a6,b6,
%		letter,
%		legal,
%		executive,
%		center,				% zentrierter druck
%		landscape,			% querformat
%		]{crop}
	 

	   
%\usepackage{anyfontsize} % hilft gegen font shape not available, was wegen \DeclareMathSizes auftreten kann
%\usepackage{relsize}                            % für größere und kleinere Gleichungen über den Befehl \mathlarger bzw. \mathsmaller, auch \textlarger und \textsmaller

