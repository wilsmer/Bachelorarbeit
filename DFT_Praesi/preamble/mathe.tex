\usepackage[%
%	    leqno,				% Nummern linksbuendig
%	    reqno,				% Nummern rechtsbuendig (Standard)
%	    fleqn,				% Gleichungen linksbuendig statt zentriert
	   ]{amsmath}
\usepackage{amsfonts}				% Um Zahlenraeume richtig darzustellen
\usepackage{mathtools}


% Betragsstriche über \abs, Doppelbetragsstriche über \norm
\DeclarePairedDelimiter\abs{\lvert}{\rvert}%
\DeclarePairedDelimiter\norm{\lVert}{\rVert}%

\usepackage[%
	    thinlines,				% dünne linien
%	    thicklines,				% dicke linien
	   ]{easybmat}				% For Matrices with dottet lines between fields, horizontal or vertical
%\usepackage{MnSymbol}				% Zusaetzliche Zeichen
\usepackage{trsym}				% Fuer Laplace-Fourier-Symbole
\usepackage{mathrsfs}				% Fuer Matheschrift
\usepackage{xfrac}
\usepackage{nicefrac}
\usepackage{esint}
%\usepackage{exscale} % lässt Klammern in Gleichungen verschwinden!!
%\DeclareMathSizes{10.95}{30}{15}{15} % 10.95 für 11pt in documentclass

\def\mathunderline#1#2{\color{#1}\underline{{\color{black}#2}}\color{black}}

\usepackage{soul}