%%%%%%%%%%%%%%%%%%%%%%%%%%%%%%%%%%%%%%%%%%%%%%%%%%%%%%%%%%%%%%%%%%%%%%%%%%%%%%%
%  Author     : Thorben Schuethe        Date    : 18.07.2012                  %
%  Filename   : newCommands.tex        Version : 2.24                         %
%-----------------------------------------------------------------------------%
%  Changelog  :                                                               %
%  10.09.2011 : build date                                                    %
%  09.08.2012 : Add more commands - for ref and cite                          %
%  21.08.2012 : Renewcommand + new environment tabular10                      %
%  01.09.2012 : Add Modulbezeichnung and proxy                                %
%  02.09.2012 : Symbolverzeichnisbefehle hinzugefügt                     v2.1 %
%  05.09.2012 : Symbolverzeichnisbefehle abgeändert                      v2.2 %
%  07.09.2012 : Neuer Befehl zum darstellen von Gui's                    v2.21%
%  12.09.2012 : Einfügen der Theoremumgebung Definition                  v2.22%
%  11.10.2012 : Einfügen des Vektorbefehls                               v2.23%
%  11.10.2012 : Einfügen neuer Befehle für Methoden                      v2.24%
%%%%%%%%%%%%%%%%%%%%%%%%%%%%%%%%%%%%%%%%%%%%%%%%%%%%%%%%%%%%%%%%%%%%%%%%%%%%%%%

%%%%%%%%%%%%%%%%%%%%%%%%%%%%%%%%%%%%%%%%
% new Commands                         %
%%%%%%%%%%%%%%%%%%%%%%%%%%%%%%%%%%%%%%%%

% COMMANDS FOR THE MASTERTHESIS
\newcommand{\theLocation}{Hamburg}      				% Ort
\newcommand{\theAuthor}{Thorben Schüthe}			% Name of the writer
\newcommand{\theTitle}{ }
\newcommand{\theTitleEn}{ }

% COMMANDS FOR SOURCE CODE IN TEXT
\newcommand{\defmethod}[1]{\textit{#1()}}			% Methoden
\newcommand{\defmethodArg}[2]{\textit{#1(#2)}}		% Methoden
\newcommand{\defstate}[1]{\textsc{#1}}				% State
\newcommand{\module}[1]{\emph{#1}}					% Modulbezeichnung
\newcommand{\proxy}[1]{\emph{#1}}					% Proxy
\newcommand{\gui}[1]{\emph{#1}}						% GUI
\newcommand{\variable}[1]{\emph{#1}}				% Variablen
\newcommand{\opsys}[1]{\emph{#1}}					% Für OS Systeme

% COMMANDS FOR A GOOD WRITING STYLE
\newcommand{\zb}{z.\,B.\ }							% Sorgt für richtige Darstellung von z.B.
\newcommand{\dH}{d.\,h.\ }							% d.h.
\newcommand{\bzw}{bzw.\ } 							% beziehungsweise
\newcommand{\ca}{ca.\ } 							% circa
\newcommand{\dgl}{dgl.\ }							% dergleichen 
\newcommand{\dsgl}{dsgl.\ }     					% desgleichen
\newcommand{\etc}{etc.\ }  							% et cetera (= und so weiter)
\newcommand{\evtl}{evtl.\ }							% eventuell f. folgende
\newcommand{\ff}{ff.\ }								% fortfolgende ggf. gegebenenfalls
\newcommand{\usw}{usw.\ }							% und so weiter 
\newcommand{\zt}{z.\,T.\ } 							% zum Teil
\newcommand{\points}{\ldots}						% besser als ..., da schrift dann anders ist
\newcommand{\separator}{.}							% Dezimaltrennzeichen
\newcommand{\vgl}{vgl.\ }							% vergleiche
\newcommand{\ggf}{ggf.\ }
\newcommand{\cplusplus}{C++}						% um C++ richtig darzustellen
\newcommand{\writeJava}{Java}						% Um JAVA richtig darzustelen


% Commands for Tables and Figures
\newcommand{\printOpt}{tp}						% Option für die Grafikposition
\newcommand{\stdWidth}{0.7\textwidth}				% Einstellen der Breite von Bildern um einheitliches Bild zu schaffen
\newcommand{\picref}[1]{Abbildung~\ref{fig:#1}}		% Verweis zu einer Abbildung ohne Seitenzahl oder ähnliches
\newcommand{\citepic}[1]{\autocite{#1}}				% Für Bilder aus anderen Quellen - So kann dies verändert werden
\newcommand{\pathOf}[1]{#1}							% Gibt den aktuellen Path zur Datei an, muss mit renewcommand für Dateien überschrieben werden

% Commands for Typesetting and Layout
\newcommand{\frontmatterOwn}{\pagenumbering{Roman}}%\renewcommand{\thepage}{\Roman{page}}}
\newcommand{\mainmatterOwn}{\setcounter{page}{1} \pagenumbering{arabic}}%\renewcommand{\thepage}{\arabic{page}}}
\newcommand{\chapterNumless}[1]{%
	\chapter*{#1}\addcontentsline{toc}{chapter}{#1}%
}
% To get the following in the table of content:
%
% Anhang
% A. erster eintrag
% B. zweiter Eintrag
%
% Im dokument selbst wird folgendes ausgegeben (auf einer Seite)
% Anhang
%
%
% A. erster eintrag
\newcommand{\firstAppendixEntry}[1]{%
	\addcontentslinetoeachtocfile{chapter}{{\large \appendixname}}
	\stepcounter{chapter}
	\addcontentslinetoeachtocfile{chapter}{\thechapter \autodot\ #1}
	\chapter*{{\Huge\appendixname} \linebreak \linebreak \thechapter\autodot\ #1}%
}
% Glossry - Symbolverzeichnis
%\newcommand{\symbolentryGreek}[4]{\glossary{sort=a#4, name={#1},description={#2}}}
%\newcommand{\symbolentryRoman}[4]{\glossary{sort=b#4, name={#1},description={#2}}}
% Nomencl - Symbolverzeichnis
%\newcommand{\symbolentryGreek}[4]{\nomenclature[a#4]{#1}{#2}}
%\newcommand{\symbolentryRoman}[4]{\nomenclature[b#4]{#1}{#2}}
% Glossries - Symbolverzeichnis
\newcommand{\symbolentryGreek}[4]{\newglossaryentry{#4}{name={#1}, parent=greekletter, description={#2}, symbol={#3} ,sort={a#4}, type=symbols}}
\newcommand{\symbolentryRoman}[4]{\newglossaryentry{#4}{name={#1}, description={#2}, symbol={#3} ,sort={b#4}, type=symbols, parent=latinletter}}

% Others
\newcommand{\todo}[1]{\textcolor{red}{\textbf{#1}}}
\newcommand{\company}[1]{\textsf{#1}}		% Zum Hervorheben von Firmennamen bei Großschreibung
\newcommand{\ST}{ \textsf{STILL\ }}			% STILL - mit guter Schreibweise
\newcommand{\tramark}[1]{\textsl{#1}}					% trademark
%\newcommand{\bigLetter}[1]{\emph{#1}}				% Zum Hervorheben von Großgeschriebenen - wie FM-X
\newcommand{\myCaptionEnd}{.}						% Wird am Ende jeder Bildunterschrift/Tabellenüberschrift gesetzt
\newcommand{\subfigref}[2]{Abbildung~\ref{fig:#1}\subref{fig:#2}} % Um auf Subfigures richtig zu verweisen
\newcommand{\tvec}[3]{\ensuremath{\left(#1, #2, #3 \right)}}
\newcommand{\okmarker}{\textcolor{green}{\Checkmark}}% Um den Haken grün zu färben
\newcommand{\nokmarker}{\textcolor{red}{\XSolidBrush}}% Um den Haken grün zu färben
%%%%%%%%%%%%%%%%%%%%%%%%%%%%%%%%%%%%%%%%
% new Environments 	                   %
%%%%%%%%%%%%%%%%%%%%%%%%%%%%%%%%%%%%%%%%
\newenvironment{tabularx10}{%
  \fontsize{10}{12}\selectfont\tabularx
}{%
  \endtabularx
}													% Tabellen in kleinerer Schrift als der Normaltext, laut Bachelor-,
													% Master- und Doktorarbeit auf Seite 137 - Tabellensatz -

%%%%%%%%%%%%%%%%%%%%%%%% Das hier verwenden%%%%%%%%%%%%%%%%%%%%%%%%%%%%%%%%%%%%%%%%%%%%%%%%%%%%%%%%%%
\newtheorem{definitionDe}{Definition}[chapter]			% Für Definitionsumgebung													
\newenvironment{fshaded}{%
\def\FrameCommand{\fcolorbox{framecolor}{shadecolor}}%
\MakeFramed {\FrameRestore}}%
{\endMakeFramed}
\newenvironment{definit}[1][]{\definecolor{shadecolor}{rgb}{0.95,0.95,0.95}%
\definecolor{framecolor}{rgb}{0,0,0}%
\begin{fshaded}\begin{definitionDe}[#1]}{\end{definitionDe}\end{fshaded}}
%%%%%%%%%%%%%%%%%%%%%%%%%%%%%						%%%%%%%%%%%%%%%%%%%%%%%%%%%%%%%%%%%%%%%%%%%%%%%%%
%%%%%%%%%%%%%%%%%%%%%%%%%%%%%						%%%%%%%%%%%%%%%%%%%%%%%%%%%%%%%%%%%%%%%%%%%%%%%%%
%%%%%%%%%%%%%%%%%%%%%%%%%%%%%	 ODER   DAS			%%%%%%%%%%%%%%%%%%%%%%%%%%%%%%%%%%%%%%%%%%%%%%%%%
%%%%%%%%%%%%%%%%%%%%%%%%%%%%%						%%%%%%%%%%%%%%%%%%%%%%%%%%%%%%%%%%%%%%%%%%%%%%%%%
%%%%%%%%%%%%%%%%%%%%%%%%%%%%%						%%%%%%%%%%%%%%%%%%%%%%%%%%%%%%%%%%%%%%%%%%%%%%%%%
%\newtheorem{definit}{Definition}[chapter]			% Für Definitionsumgebung
%%%%%%%%%%%%%%%%%%%%%%%%%%%%%%%%%%%%%%%%%%%%%%%%%%%%%%%%%%%%%%%%%%%%%%%%%%%%%%%%%%%%%%%%%%%%%%%%%%%%%
%%%%%%%%%%%%%%%%%%%%%%%%%%%%%%%%%%%%%%%%
% overwrite Commands                   %
%%%%%%%%%%%%%%%%%%%%%%%%%%%%%%%%%%%%%%%%
\renewcommand{\mkcitation}[1]{ \autocite{#1}}		% Not to have the () brackets around the [] brackets of the quotation!
\renewcommand{\mktextquote}[6]{#1#2#3#6#4#5}
%Den Punkt am Ende jeder Beschreibung deaktivieren
\renewcommand*{\glspostdescription}{}
%Style Symbolverzeichnis definieren 
\newglossarystyle{symbver}{ % put the glossary in a longtable environment: 
\renewenvironment{theglossary} 
{\begin{longtable}{lp{\glsdescwidth}ccp{\glspagelistwidth}}} 
{\end{longtable}} 
\renewcommand*{\glossaryheader}{} 
\renewcommand*{\glossaryentryfield}[4]{ 
\glstarget{##1}\\[0.1cm]{##2}&{##3} {##4}}% \\[0.5cm] Zeilenabstand zwischen Einträgen 
\renewcommand*{\glossarysubentryfield}[6]{ 
\glossaryentryfield{##2}{##3}{##4}{##5}} 
\renewcommand*{\glsgroupskip}{}}

\makeatletter
\newglossarystyle{myindex}{%
  \renewenvironment{theglossary}%
    {\setlength{\parindent}{0pt}%
     \setlength{\parskip}{0pt plus 0.3pt}%
     \let\item\@idxitem}%
    {}%
  \renewcommand*{\glossaryheader}{}%
  \renewcommand*{\glsgroupheading}[1]{}%
\renewcommand*{\glossaryentryfield}[5]{%
\item\glstarget{##1}{##2}% ##1 ?? ##2 = obergruppe
  \ifx\relax##4\relax% ##4 ist ne breite
  \else
    \space(##4)%
  \fi
  ##3\glspostdescription \space ##5}%
  \renewcommand*{\glossarysubentryfield}[6]{%
    \ifcase##1\relax
      % level 0
      \item
    \or
      % level 1
      \subitem
    \else
      % all other levels
      \subsubitem
    \fi
    \glstarget{##2}{##3}%
    \ifx\relax##5\relax
    \else
      \space(##5)%
    \fi
    \space##4\glspostdescription\space ##6}%
  \renewcommand*{\glsgroupskip}{\indexspace}
    \renewcommand*{\glsgroupheading}[1]{%
    \item\textbf{\glsgetgrouptitle{##1}}\indexspace}%
  }
  
\newglossarystyle{mylist}{%  
 % put the glossary in the itemize environment:  
 \renewenvironment{theglossary}%  
   {\begin{itemize}}{\end{itemize}}%  
 % have nothing after \begin{theglossary}:  
 \renewcommand*{\glossaryheader}{}%  
 % have nothing between glossary groups:  
 \renewcommand*{\glsgroupheading}[1]{}%  
 \renewcommand*{\glsgroupskip}{}%  
 % set how each entry should appear:  
 \renewcommand*{\glossaryentryfield}[5]{%  
 \item % bullet point  
 \glstarget{##1}{##2}% the entry name  
 \space (##4)% the symbol in brackets  
 \space ##3% the description  
 \space [##5]% the number list in square brackets  
 }%  
 % set how sub-entries appear:  
 \renewcommand*{\glossarysubentryfield}[6]{%  
   \glossaryentryfield{##2}{##3}{##4}{##5}{##6}}%  
 } 

\newglossarystyle{mysuper}{%
  \renewenvironment{theglossary}%
    {\tablehead{}\tabletail{}%
     \begin{supertabular}{lp{0.5\glsdescwidth}}}%
    {\end{supertabular}}%
  \renewcommand*{\glossaryheader}{}%
  \renewcommand*{\glsgroupheading}[1]{}%
  \renewcommand*{\glossaryentryfield}[5]{%
    \glsentryitem{##1}\glstarget{##1}{##2} & ##3\glspostdescription\space ##5\\}%
  \renewcommand*{\glossarysubentryfield}[6]{%
     &
     \glssubentryitem{##2}%
     \glstarget{##2}{\strut}##4\glspostdescription\space ##6\\}%
  \renewcommand*{\glsgroupskip}{ & \\}%
}  
  
\makeatother

\newcommand{\abstractentry}[2]{
%	\textbf{\large#1}\\ 
	\subsubsection*{#1}
	\nobreakspace 
	\begin{tabular}{lp{142mm}}
		\hspace*{3mm} & #2 \\
	\end{tabular}
	\vfill%
}

%C++
\newcommand\Cpp{C\nolinebreak[4]\hspace{-.05em}\raisebox{.4ex}{\relsize{-3}{\textbf{++}}}}

%\usepackage{etoolbox}% http://ctan.org/pkg/etoolbox
%\makeatletter
%\patchcmd{\lst@GLI@}% <command>
%  {\def\lst@firstline{#1\relax}}% <search>
%  {\def\lst@firstline{#1\relax}\def\lst@firstnumber{#1\relax}}% <replace>
%  {\typeout{listings firstnumber=firstline}}% <success>
%  {\typeout{listings firstnumber not set}}% <failure>
%\makeatother
