\usepackage[%
%	    leqno,				% Nummern linksbuendig
%	    reqno,				% Nummern rechtsbuendig (Standard)
%	    fleqn,				% Gleichungen linksbuendig statt zentriert
	   ]{amsmath}
\usepackage{amsfonts}				% Um Zahlenraeume richtig darzustellen
\usepackage{mathtools}
\usepackage{amssymb}

% Betragsstriche über \abs, Doppelbetragsstriche über \norm
\DeclarePairedDelimiter\abs{\lvert}{\rvert}%
\DeclarePairedDelimiter\norm{\lVert}{\rVert}%

\usepackage[%
	    thinlines,				% dünne linien
%	    thicklines,				% dicke linien
	   ]{easybmat}				% For Matrices with dottet lines between fields, horizontal or vertical
%\usepackage{MnSymbol}				% Zusaetzliche Zeichen
\usepackage{trsym}				% Fuer Laplace-Fourier-Symbole
\usepackage{mathrsfs}				% Fuer Matheschrift
\usepackage{xfrac}
\usepackage{nicefrac}
\usepackage{esint}
%\usepackage{exscale} % lässt Klammern in Gleichungen verschwinden!!
%\DeclareMathSizes{10.95}{30}{15}{15} % 10.95 für 11pt in documentclass

\def\mathunderline#1#2{\color{#1}\underline{{\color{black}#2}}\color{black}}

\usepackage{soul}

\usepackage[%				
%	    group-separator 	= {.}, 		% group-seperator={,} -> 1,345,234.23
%   	    round-mode 		= places,		% round-mode= places(2digits), figures(1digit)
%	    round-precision 	= 3,		% round-precision= x , xdigits are rounded to
	    binary-units 	= true,		% Laden von \byte \bit, \kibi usw. (false default)
	    locale 		= DE,		% lacale= DE, uses german
	    per			= slash,
	    alsoload            = binary,
%	    loctolang		={
%				  UK:english, 
%				  DE:ngerman
%		      		 },		% loctolang={USA:USenglish,DE:ngerman}, if language changed in document
	    detect-all				% detect-all : uses the text options not the math mode to set the letters/numbers
	   ]{siunitx}
