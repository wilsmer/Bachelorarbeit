\usepackage[  % Packet ist jetzt glossaries-extra statt glossaries - keine Ahnung, ob die es die auskommentierten Optionen gibt!
	    xindy,%={language=german-modern, codepage=utf8},
	    automake,				
	    numberedsection,
%	    toc,      				% Fügt Eintrag ins Inhaltsverzeichnis hinzu, defualt false
	    nonumberlist,			% keine Ausgabe der Seitenzahlen
%	    nowarn,				% unterdrückt alle warungen des pakets
%	    nomain,				% kein hauptglossary wird erzeugt, acronym oder newglossary um Einträge zu erzeugen
%	    sanitize={%				% unterdrückt die keys für einträge oder fügt sie hinzu
%	    name        = false,%
%	    description = false,%
%	    symbol      = false},%
%	    translate = true,			% zur übersetzung von einträgen - siehe doku, default false
	    hyperfirst = true,			% erzeugt hyperlink zum glossar bei erster verwendung, default true
%	    hyper = false,			% Hyperlinks ausschalten
%	    numberline,				% Eintrag ins toc mit Nummerierung
%	    section=section,			% in welcher ebene der eintrag, default chapter if loaded sonst section
%	    style = long,			% default list, sonst: long, super, tree
%	    nolong,				% pakete für longtable werden nicht geladen und befehle nicht definiert
%	    nosuper,				% supertabular wird nicht geladen, pakete nicht definiert
%	    nolist,				% list wird nicht geladen definiert
%	    notree,				% gloassrytree wird nicht geladen definiert
%	    nostyles,				% keine styles werden geladen, eigene definieren
%	    counter = page,			% page default, or any other counter
%	    sort = def,				% def - sortiert nach definition, standard(default)-sortiert nach sort key, sonst name, use - sortieren nach reihenfolge wie sie im dokument vorkommen
%	Acronym options
%	    acronym,  				% Seperate Liste für Acronyme zum Glossary, default - false
%	    acronymlist={},			% wenn mehr als eine abkürzungsliste, muss diese hier mit angegeben werden damit das glossary als acronym behandelt wird
%	    description,			% erlaubt eine zusätzliche beschreibung
%	    footnote,				% schreibt die long version als fußnote bei erster nutzung
%	    smallcaps,				% acronyme werden als kapitelchen angezeigt
%	    smaller,				% kleinere schrift, laden von relsize für befehl textsmaller erforderlich
%	    dua,				% langform wird immer ausgegeben
	    shortcuts				% um selbe befehle wie acronym paket zu haben
	   ]{glossaries-extra}		


\GlsSetXdyCodePage{duden-utf8}

% sorgt dafür, dass die Erklärungen alle gleich eingerückt sind
\setglossarystyle{long}
\renewcommand{\glsnamefont}[1]{\textbf{#1}}

% Bachlorarbeit.gls erzeugen:
% $ cd Bachelorarbeit
% $ makeglossaries "Bachelorarbeit"

\makeglossaries 
