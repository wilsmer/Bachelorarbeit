% load the document class KOMA-Script Report
\documentclass[	
		a4paper,			% Papierformat waehlen
		12pt,				% Schriftgroesse
%		BCOR5mm,			% Bindungskorrektur
%		draft,				% Randueberschreitungen werden schwarz markiert
%		DIV13,				% Unterteilung des Blatts in DIVxx - xx Teile              OBSOLETE
						% Je kleiner die Zahl umso groesser die Raender
						% Um Optimale Anpassung zu erhalten : DIVcalc + \typearea...
%		BCOR5mm,			% Bindekorrektur BCORXmm - Xmm breit                       OBSOLETE
		%pointlessnumbers,		% KEIN ZWEITER PUNKT IN DER NUMMERIERUNG DER ÜBERSCHRIFTET 2.2 statt 2.2.	
%
%      KOPF/FUSSZEILE
%		headinclude,			% Head gehoert mit zum Textblock/auskommentiert im Seitenrand
		headsepline,			% Trennlinie zwischen Kopf und Text, schaltet automatisch 
						% headinclude mit ein
%		footinclude,			% wie Head nur Foot
%		footsepline,			% wie headsepline
%
%      LAYOUTPARAMETER
%		twocolumn,			% Zweispaltige Texte
		onecolumn,			% Einspaltig
%		twoside,			% Zweiseitiger Druck re -li sind gespiegelt vom Satzspiegel
		oneside,			% Einseitig
%		openright,			% Sollte bei zweiseitigem Druck eignestellt werden
%		cleardoublestandard		% Alles fuer 2Seitige mit openright: Kolumnentitel linke Seite
%		cleardoubleplain		% nur Seitenzahlen linke Seite
%		cleardoubleempty		% linke Seite ganz leer
%		chapterprefix,			% Kapitel wir zu beginn eines Kapitels ausgegeben
%		appendixprefix,			% Anhang wird vor Kapiteln im Anhang ausgegeben
		headings=normal,		% kleine ueberschriften, oder small-, oder big-headings.
						% Standard ist Big
%
%	TABELLENEIGENSCHAFTEN
		captions=tableheading,		% Um richtigen Abstand fuer Ueberschrift zu erhalten, 
						% Ueberschrift wird genommen
%
%	VERZEICHNISEINSTELLUNGEN
%		tocleft,			% Setzt das Inhaltverzeichnis nicht eingerueckt nach links
%		liststotoc,			% Abb.- und Tab.Verzeichnisse ins Inhaltsverzeichnis
%		liststotocnumbered, 		% wie liststotoc nur mit Gliederungspunktangabe
%		bibtotoc,			% Literaturverzeichnis ins Inhaltverzeichnis
%		bibtotocnumbered,		% s.o. nur mit Gliederungsebene
%		idxtotoc,			% Eintrag ins Inhaltsverzeichnis fue Index
%		openbib,			% Veraendert aussehen des Lit.-Verzeichnisses
%	Language
%		ngerman,			% to pass as standard to all packages
%halfparskip,                                         % Europäischer Satz: Abstand zwischen Absätzen
%	abstracton,																					 % Spezielle Formatierung, die erlaubt, dass die
																											 % Zusammenfassung vor dem Inhaltsverzeichnis steht
	%draft,		 % Es handelt sich um eine Vorabversion	
%	final		 % Es handelt sich um die endgültige Version
]{scrreprt}

% load packages for european, espacially german users

\usepackage[T1]{fontenc}			% Schriften fuer Europaeische Zeichen passend Kodiert
\usepackage[utf8]{inputenc}        		% Eingabe von Umlauten, ss usw. - 
						% utf8: kann nicht jeder Editor speicher, aber plattformuebergreifend
						% latin1: Unix, VMS, Windows
						% ansinew: Windows
						% latin9: wie latin1, jedoch mit Eurozeichen
						
\usepackage[					% Passt Konventionen an Sprache an
	     english, 
	     ngerman				% ngerman: Neue Deutsche Rechtschreibung
	    ]{babel}				% z.B. UKenglish, USenglish, canadian, german, austrian, naustrian
	    
%\usepackage[latin1]{inputenc}                   % direkte Umlauteingabe (ä statt "a)
                                                % latin1/latin9 für unixoide Systeme
                                                % (latin1 ist auch unter Win verwendbar)
                                                % ansinew für Windows
                                                % applemac Macs
                                                % cp850 OS/2
\usepackage{times}              	        % Schriften Paket
\usepackage{array,ragged2e} 			% Wichtig für Abstandsformatierung
%---------------------------------------------------------------------------------------------------
\usepackage{cmbright}                                  % serifenlose Schrift als Standard
                                                       % + alle für TeX benötigten mathematischen
                                                       %   Schriften einschließlich der AMS-Symbole
\usepackage[scaled=.90]{helvet}                        % skalierte Helvetica als \sfdefault
\usepackage{courier}                                   % Courier als \ttdefault
%---------------------------------------------------------------------------------------------------
%\usepackage[automark]{scrpage2}                        % Anpassung der Kopf- und Fußzeilen
\usepackage{xspace}                                    % Korrekter Leerraum nach Befehlsdefinitionen
\usepackage{setspace}				  % Dieses Package brauchen wir für den 				
\usepackage[pdftex]{graphicx}
\usepackage[absolute,overlay]{textpos}         

%\usepackage{natbib}                                    % Neuimplementierung des \cite-Kommandos
\usepackage{bibgerm}       				 % Deutsche Bezeichnungen
\usepackage[final]{pdfpages}                           % include pages of external PDF documents

%\usepackage{makeidx}					% Paket zur Erstellung eines Stichwortverzeichnisses
%\makeindex						 % Automatische Erstellung des Stichwortverzeichnis
%\usepackage[intoc,
%						german,
%						prefix]{nomencl}
%\makenomenclature
	    
%\hyphenation{}					% Trennung von Woertern falls nicht automatisch korrekt erkannt
%\usepackage{icomma}				% Bei deutscher Schreibweise, damit keine

%\usepackage[scaled=0.66]{luximono}		% Schrift fuer Typewriter - Quellcode ~8pt
				

% Fuer optimale Anpassung des Satzspiegels
%\typearea[current]{calc}			% Bestimmt Rasterzahl aus Schriftgroesse und Anzahl der 		
						% Zeichen/Woerter pro Zeile

% FUSS UND KOPFZEILENANPASSUNG
%\pagestyle{plain}				% empty: Keine Kopf/Fusszeile
						% plain: Seitenzahl am Fussende
						% headings: aktiviert lebende Kolumnentitel (Kapitel im Kopf)
						% myheadings : eigene Kopf- fußzeilen
						% fancy: Erlaubt die Verwendung der in dem Paket "fancyhdr" 
						% definierten Befehle zur Erstellung eigener Kopf- und Fußzeilen

%%error
% scrpage2 ist obsolete!
%\usepackage[
%	    automark				% takes the chapter variant
%	   ]{scrpage2}				% Fuer eigene Kopf/Fusszeilen, ermoeglicht. 
% stattdessen scrlayer-scrpage:

\usepackage{scrlayer-scrpage}
\pagestyle{scrheadings} %, scrplain}
\clearscrheadings				% clear old header style
\clearscrplain					% clear plain header style
\clearscrheadfoot				% clear foot style
\automark[section]{chapter}
\ohead{\pagemark}				% beide Außenseiten des Headers --Seitenzahl
\ihead{\headmark}				% beide Innenseiten Headers -- Chapter

%\chead						% zentriert 
%\lehead					% left even
%\cehead
%\rehead
%\lohead					% left odd
%\cohead
%\rohead
%%


%\lehead{\leftmark}
%\lohead{\rightmark}
%\ofoot[\pagemark]{}				% auf plain seiten Seitenzahl außen
									
% Schriftfamilien Laden
\usepackage{lmodern}				% Laedt die Schriftart Latin-Modern fuer Text
\usepackage{textcomp}
%\usepackage{courier}
%\rmfamily
%\sffamily
%\ttfamily
%\renewcommand{\rmdefault}{
%			  pbk	% Bookman
%			  phv	% Helvetica
%			  cmr	% CM Roman
%			  ppl	% Palatino
%			  ptm	% Times Roman
%			  pag	% Avant Garde
%			  pcz	% Zapf Chancery
%			 }
%\renewcommand{\familydefault}{\sfdefault}

% Alternativ Schriftart (Textshchrift) ändern über usepackage:
%\usepackage{
%	    mathpazo	% Palatino
%	    mathptmx	% Times
%	    avant	% Avant Garde
%	    courier	% Courier
%	    chancery	% Zapf Chancery
%	    bookman	% Bookman
%	    newcent	% New Century Schoolbook
%	    charter	% Charter
%	    helvet	% Helvetica
%} 
%\usepackage[scaled=0.92]{helvet}		% Helvetica, ist größer als Times und sollte bei verwendung beider skaliert werden.

% Im Dokument wechslen:
%\fontfamily{pbk}\selectfont

\usepackage[
	    babel,
%	    german=guillemets,  		% franz. <<  >>
	    german= quotes,			% deutsche Anfuehrungszeichen "
	   ]{csquotes}				% Laden der richtigen Anfuehrungszeicehn fuer deutsche sprache
						% = quotes fuer englische sprache



	   
%\usepackage{setspace}				% Fuer Abstaende im Dokument
%\onehalfspacing				% aendert Zeilenabstand auf 1 1/2
									
%\usepackage{microtype}				% optischer Randausgleich in PDF's
									% ungeeignet fuer internettexte o.ae.	
	   
%\usepackage[					% Papierformate auf denen gedruckt werden soll
%		a0,b0,
%		a1,b1,
%		a2,b2,
%		a3,b3,
%		a4,b4,
%		a5,b5,
%		a6,b6,
%		letter,
%		legal,
%		executive,
%		center,				% zentrierter druck
%		landscape,			% querformat
%		]{crop}
	 
									
\usepackage[a4paper]{geometry}			% Anpassung von Seitenraender per Hand
\geometry{					% Wenn Zweiseitig-> left->inner : right->outer	
	    top=2cm, 				% Weitere Moeglichkeiten: height - Texthoehe, width - Textbreite
	    bottom=5cm,
	    inner=2cm,
	    outer=3cm
         }	

\setlength{\headheight}{1.5cm}
\setlength{\voffset}{0.5cm}


\usepackage[					% To set Text at a specific 
		absolute,			% absolute - absolute position on the page
		overlay,			% overlay - if using absolute option text is placed below other things
%		showboxes,			% showboxes - shows boxes around the text
%		noshowtext,			% noshowtext - just show the box if it is on
%		verbose,			% verbose - package writing things to output like calculations
%		quiet,				% quiet turns this off : verbose = default
	    ]{textpos}

\usepackage[
	    hyperref 	= false,		% switch off hyperref hack
	    float 	= true,			% switch off float hack
	    listings 	= true,  		% switch off listings hack
	   ]{scrhack}				% to get rid off the warning @addtocbasic

	   
%\usepackage{anyfontsize} % hilft gegen font shape not available, was wegen \DeclareMathSizes auftreten kann
\usepackage{relsize}                            % für größere und kleinere Gleichungen über den Befehl \mathlarger bzw. \mathsmaller, auch \textlarger und \textsmaller

\usepackage{lscape}

%---------------------------------------------------------------------------------------------------
% Anpassung der Parameter, die TeX bei der Berechnung der Zeilenumbrüche verwendet:
%---------------------------------------------------------------------------------------------------
\tolerance 1414
\hbadness 1414
\emergencystretch 1.5em
\hfuzz 0.3pt
\widowpenalty=10000
\vfuzz \hfuzz
\raggedbottom