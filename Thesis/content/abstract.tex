%---------------------------------------------------------------------------------------------------
% Erstellung von Titelblatt (Seite 2) 
% Anwendung:
% \createAbstract{Art der Arbeit}{Typ der Arbeit}{Author}{Titel}{Titel Englisch}{Stichworte}{Keywords}{Zusammenfassung}{Abstract}
%---------------------------------------------------------------------------------------------------
	\thispagestyle{empty}
	\pdfbookmark[1]{Kurzzusammenfassung}{Kurzzusammenfassung}
	%\enlargethispage{1cm} % 2cm zuvor

	%\begin{small}
{
%\makeatletter
%%Größe von chapter direkt ändern:
%   \renewcommand*{\size@section}{\small}%
%\makeatother
	\subsection*{Thomas Lattmann}%
%
	\subsubsection*{Thema der Bachelorarbeit}
	\par \begingroup \leftskip=1cm % Parameter anpassen
	\noindent {Chipimplementation einer zweidimensionalen Fouriertransformation für die Auswertung eines Sensor-Arrays}%
	\par \endgroup
	\subsubsection*{Stichworte}
	\par \begingroup \leftskip=1cm % Parameter anpassen
	\noindent {Cadence, ASIC, zweidimensionale diskrete Fouiertransformation (2D-DFT), Sensor-Array}%
	\par \endgroup
	\subsubsection*{Kurzzusammenfassung}
	\par \begingroup \leftskip=1cm % Parameter anpassen
	\noindent{In der Bachelorarbeit wird eine 8x8 zweidimensionale Fourier Transformation in VHDL für den Einsatz als Teilmodul auf einem ASIC entwickelt. Dabei wird das Chipdesign-Tool Cadence verwendet. Die Transformation wird durch eine Matrixmultiplikation realisiert und hinsichtlich Taktzyklen und Flächenbedarf optimiert, sodass ein minimaler Aufwand erforderlich ist. Ferner werden Tests zur Funktionalität  durchgeführt und der Floorplan erstellt. }%
	\par \endgroup
%		
	\vspace{1.1cm} % default 1.5
	\selectlanguage{english}%
	\subsection*{Thomas Lattmann}%
%	
	\subsubsection*{Title of the bachelor thesis}
	\par \begingroup \leftskip=1cm % Parameter anpassen
	\noindent {Chip implementation of a two dimensional Fourier transform for the analysis of a sensor array}%
	\par \endgroup
	\subsubsection*{Keywords}
	\par \begingroup \leftskip=1cm % Parameter anpassen
	\noindent {%
	Cadence, ASIC, two dimensional discrete Fourier transform (2D-DFT), sensor array
	}%
	\par \endgroup
	\subsubsection*{Abstract}
	\par \begingroup \leftskip=1cm % Parameter anpassen
	\noindent {%
		In this bachelor thesis a 8x8 twodimensional Fourier transform is developed in VHDL as a partial modul for the usage on an ASIC.
		For the implementation the chipdesign tool Cadence is used. The transform is realized as a matrix multiplication and optimized in terms of clock cycles and required area.
		Functionality test are executed and the floorplan is  created. }
	\par \endgroup
	\selectlanguage{ngerman}
}
%\end{small}