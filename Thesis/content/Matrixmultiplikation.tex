\subsection{Matrixmultiplikation}\label{sec:Matrixmultiplikation}

% Betrachtet wird zunächst die Multiplikation einer Matrix mit einem Vektor.
% 
% \begingroup
% \renewcommand*{\arraystretch}{1.0}
% 
%  \[
%    \begin{bmatrix}
%     \myBlackBox 	& \myBlackBox 		& \myBlackBox 		& \myBlackBox \\
%     \myLightgrayBox 	& \myLightgrayBox 	& \myLightgrayBox 	& \myLightgrayBox \\
%     \myLightgrayBox 	& \myLightgrayBox 	& \myLightgrayBox 	& \myLightgrayBox \\
%     \myLightgrayBox 	& \myLightgrayBox 	& \myLightgrayBox 	& \myLightgrayBox
%    \end{bmatrix}
%   \cdot
%    \begin{bmatrix}
%     \myBlackBox \\
%     \myBlackBox \\
%     \myBlackBox \\
%     \myBlackBox
%    \end{bmatrix}
%   =
%   \begin{bmatrix}%
%    \myBlackBox \\
%    \myLightgrayBox \\
%    \myLightgrayBox \\
%    \myLightgrayBox
%   \end{bmatrix}
%  \]
% \endgroup
% 
%  \vspace{1cm}  
 
 


Um nachfolgende Abschnitte besser erörten zu können, soll zunächst die Matrixmultiplikation besprochen werden.
Wie in Abbildung \ref{fig:grafikMatrizenmultiplikation} verdeutlicht, wird Element$(i,j)$ der Ergebnismatrix dadurch berechnet, dass die Elemente$(i,k)$ einer Zeile der 1. Matrix
mit den Elementn$(k,j)$ aus der zweiten Matrix multipliziert und die Werte aufsummiert werden. $i$ und $j$ sind für die Berechnung eines Elements konstant, während $k$ über alle
Elemente einer Zeile bzw. Spalte läuft.

\begin{center}
 \begin{figure}[ht!]
 \centering
\begin{minipage}{0.2\textwidth}
 \begingroup
 \renewcommand*{\arraystretch}{1.1} % Zeilenabstand
 \renewcommand*{\arraycolsep}{0.6pt} % Spaltenabstand

 \[
    \begin{bmatrix}
    \tikzmark{varrowtopleft} \myBlackBox  	& \myBlackBox 		& \myBlackBox 		& \tikzmark{varrowtopright} \myBlackBox \\
                             \myLightgrayBox 	& \myLightgrayBox 	& \myLightgrayBox 	& \myLightgrayBox \\
                             \myLightgrayBox 	& \myLightgrayBox	& \myLightgrayBox	& \myLightgrayBox \\
    \tikzmark{varrowbottom}  \myLightgrayBox 	& \myLightgrayBox 	& \myLightgrayBox 	& \myLightgrayBox 
   \end{bmatrix}
 \]
 \endgroup
  \tikz[overlay,remember picture] {
  \draw[->] ([yshift=1.5ex,xshift=-2ex]varrowtopleft) -- ([xshift=-2ex]varrowbottom)
            node[midway,left] {$i$};
  \draw[->] ([xshift=2ex,yshift=4ex]varrowtopleft) -- ([yshift=4ex]varrowtopright)
            node[midway,above] {$k$};
}
\end{minipage}
\begin{minipage}{0.1\textwidth}
 \hspace{-.5cm}
 \[
  \cdot
 \]
\end{minipage}
\begin{minipage}{0.2\textwidth}
 \begingroup
 \renewcommand*{\arraystretch}{1.1} % Zeilenabstand
 \renewcommand*{\arraycolsep}{0.6pt} % Spaltenabstand
 \[
   \begin{bmatrix}
    \tikzmark{varrowtopleft} \myLightgrayBoxHigh & \myBlackBoxHigh & \myLightgrayBoxHigh & \tikzmark{varrowtopright} \myLightgrayBoxHigh \\
                             \myLightgrayBoxHigh & \myBlackBoxHigh & \myLightgrayBoxHigh & \myLightgrayBoxHigh \\
                             \myLightgrayBoxHigh & \myBlackBoxHigh & \myLightgrayBoxHigh & \myLightgrayBoxHigh \\
    \tikzmark{varrowbottom}  \myLightgrayBoxHigh & \myBlackBoxHigh & \myLightgrayBoxHigh & \myLightgrayBoxHigh 
   \end{bmatrix}
 \]
 \endgroup
   \tikz[overlay,remember picture] {
  \draw[->] ([yshift=1.5ex,xshift=-2ex]varrowtopleft) -- ([xshift=-2ex]varrowbottom)
            node[midway,left] {$k$};
  \draw[->] ([xshift=2ex,yshift=4.5ex]varrowtopleft) -- ([yshift=4.5ex]varrowtopright)
            node[midway,above] {$j$};
}
\end{minipage}
\begin{minipage}{0.05\textwidth}
 \[
  =
 \]
\end{minipage}
\begin{minipage}{0.3\textwidth}
\begingroup
\renewcommand*{\arraystretch}{1.1} % Zeilenabstand
\renewcommand*{\arraycolsep}{0.8pt} % Spaltenabstand
\begin{align*}
   \begin{bmatrix}
    \tikzmark{varrowtopleft} \myLightgrayBox 	& \myBlackBox		& \myLightgrayBox 	& \tikzmark{varrowtopright} \myLightgrayBox \\
                             \myLightgrayBox 	& \myLightgrayBox 	& \myLightgrayBox 	& \myLightgrayBox \\
                             \myLightgrayBox 	& \myLightgrayBox 	& \myLightgrayBox 	& \myLightgrayBox \\
    \tikzmark{varrowbottom}  \myLightgrayBox 	& \myLightgrayBox 	& \myLightgrayBox 	& \myLightgrayBox 
   \end{bmatrix}
 \end{align*} 
 \endgroup
    \tikz[overlay,remember picture] {
  \draw[->] ([yshift=1.5ex,xshift=-2ex]varrowtopleft) -- ([xshift=-2ex]varrowbottom)
            node[midway,left] {$i$};
  \draw[->] ([xshift=2ex,yshift=4ex]varrowtopleft) -- ([yshift=4ex]varrowtopright)
            node[midway,above] {$j$};
}
\end{minipage}

\caption{Veranschaulichung der Matrixmultiplikation.}
\label{fig:grafikMatrizenmultiplikation}
\end{figure}
\end{center}