\section{Binäre Zahlendarstellung von Festkommazahlen}

\subsection{Integer im 1er-Komplement}
Bei der Interpretation des Bitvektors als Integer im Einerkomplement werden die Bits anhand ihrer Position im Bitvektor gewichtet, wobei as niederwertigste Bit 
(LSB, least significant bit) dem Wert für den Faktor $2^0$ entspricht, das Bit links davon dem für $2^1$ und so weiter. Die Summe aller Bits, ohne das höchstwertigste, 
multipliziert mit ihrer Wertigkeit (Potenz) ergibt den Betrag der Dezimalzahl. Das höchstwertigste Bit (MSB, most significant bit) gibt Auskunft darüber, ob es sich 
um eine negative oder positive Zahl handelt. Dies hat zur Folge, dass es eine positive und eine negative Null und somit eine Doppeldeutigkeit gibt. Desweiteren wird
ein LSB an Auflösung verschenkt. Der Wertebereich erstreckt sich von $-2^{MSB-1}+1\,LSB$ bis $2^{MSB-1}-1\,LSB$

Diese Darstellung hat den Vorteil, dass sich das Ergebnis einer Multiplikation der Zahlen $a \cdot b$ und $-a \cdot b$ nur im vorderste Bit unterscheidet. Darüber hinaus
lässt sich das Vorzeichen des Ergebnisses durch eine einfache XOR-Verknüpfung der beiden MSB der Multiplikanden ermitteln. 
Die eigentliche Multiplikation beschränkt sich auf die Bits MSB-1 bis LSB.
Da als einziger konstanter Multiplikand in der 8x8-DFT-Matrix der Faktor $\pm\frac{\sqrt{2}}{2}$ auftaucht, also das oben angeführte Beispiel zutrifft, erschien diese
Darstellungsform zwischenzeitlich interessant.

Nachteile zeigen sich hingegen bei der Addition sowie Subtraktion negativer Zahlen. Auch hierfür gibt es schematische Rechenregeln, diese erfordern jedoch mehr 
Zwischenschritte als im Zweierkomplement. Darüberhinaus ist dieses Verfahren aufgrund der geringen Bedeutung in keiner VHDL-Bibliothek implementiert.
(Verifizieren!)


\subsection{Integer im 2er-Komplement}
Bei der Interpretation als Zweierkomplement kann anhand es MSB ebenfalls erkannt werden, ob es sich um eine positive oder negative Zahl handelt. Dennoch wird es nicht
als Vorzeichenbit gewertet. Viel mehr bedeutet ein gesetztes MSB $-2^{MSB-1}$, welches der negativsten darstellbaren Zahl entspricht. Hierbei sind alle anderen 
Bits auf 0. Für gesetzte Bits wird der Dezimalwert, wie beim Einerkomplement beschrieben, berechnet und auf den negativen Wert aufaddiert. Wenn das MSB nicht gesetzt
ist, wird der errechnete Dezimalwert auf 0 addiert. Auf diese Weise lassen sich Zahlen im Wertebereich von $-2^{MSB-1}$ bis $2^{MSB-1}-1 \,LSB$ darstellen. Der positive
Wertebereich ist also um ein LSB kleiner als der negative und es gibt keine doppelte Null.

Um das Vorzeichen umzukehren müssen alle Bits invertiert werden. Auf den neuen Wert muss abschließend 1 LSB addiert werden.

Vorteile bei dieser Darstellung ist, dass die mathematischen Operationen Addition, Subtraktion und Multiplikation direkt angewand werden können. Unterstützt werden sie z.B. 
von den Datentypen \texttt{unsigned} sowie \texttt{signed}, welche in der Bibliothek u.a. \texttt{ieee.numeric\_std.all} definiert sind.




\subsection{SQ-Format im 2er-Komplement}

Im SQ-Format werden Zahlen als vorzeichenbehafteter Quotient (signed quotient) dargestellt. Die konkretere Schreibweise von beispielsweise S1Q10 bedeutet, dass zusätzlich zu
einem Vorzeichenbit noch ein weiteres Bit vor dem Komma steht. Für den Quotient stehen 10 Bit zur Verfügung, was einer maximalen Auflösung  von 
$1\,LSB = 2^{-10} = \frac{1}{1024} = 9,765625\cdot10^{-4}$ entspricht. Der Wertebereich liegt in diesem Fall liegt bei $-2$ bis $1,999023438$. Er wurde in der vorliegenden 
Arbeit so gewählt, da sich hiermit die Werte $\pm\nicefrac{\SI{3,3}{\V}}{2} = \SI{\pm1,65}{\V}$ darstellen lassen, was nach Abzug des Offsets den Eingangsspannungen des \gls{adc}
von $\SI{0}{\V}$ bis $\SI{3,3}{\V}$ entspricht und zum derzeitigen Stand des Projekts davon ausgegangen wird, dass der verwendete \gls{adc} Werte mit zwölf Bit Breite ausgibt. 
Es wird von einer Vorverarbeitung ausgegangen, die dies erledigt.




\section{Auswirkungen der Bitbegrenzung}
\subsection{Maximale Auflösung}

Um einen guten Kompromiss aus ausreichender Genauigkeit, Geschwindigkeit und Platzbedarf zu erzielen, wird von Eingangs- / Ausgangssignalen mit 12 Bit Breite zwischen
den einzelnen Komponenten auf dem Chip ausgegangen.

Sicherlich ist eine hohe Genauigkeit erstrebenswert. Es gilt jedoch zu bedenken, dass mit höheren Bitbreiten auch der Platzbedarf jedes einzelnen Datensignals aufgrund der 
zusätzlich benötigten Leitungen sowie der Flip-Flops für die (Zwischen-) Speicherung, linear steigt. 
Bei Additionen und insbesondere Multiplikationen geht mit jedem zusätzlichen Bit ebenfalls ein linear steigender Zeitbedarf einher.
Eine Bitbreite von größer 24 Bit (bei Eingangsspannungen kleiner \SI{5}{V}) ist darüber hinaus bei \gls{adc} nicht sinnvoll, da durch thermisches Rauschen die ermittelten 
Werte beeinflusst werden und die Pegel des Rauschen in dieser Größenordnung liegen.
Derzeit wird davon ausgegangen, dass der Chip in einer Strukturgröße von \SI{350}{nm} gefertigt wird, sodass sich jeder zusätzliche Platzbedarf merklich auswirkt.

Akkumulation von Fehlern kurz ansprechen.
 
\subsection{Rauschen}

