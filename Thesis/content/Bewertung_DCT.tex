\section{Bewertung verschiedener DFT- und DCT-Größen}
In diesem Abschnitt sollen Erkenntnisse gewonnen werden, auf denen basierend später die Wahl der Größe der Transformationsmatrix getroffen werden kann.
\subsection{Bewertung verschiedener DCT-Größen}
In Tabelle \ref{tab:DCT-TwiddlefaktorMatrizenBewertung} ist die Gegenüberstellung der genannten Größen zu sehen. Für die Bewertung wurd das 
Matlab-Skript aus Anhang \ref{src:dct_bewertung} geschrieben.
Ersichtlich ist, dass die Anzahl verschiedener nicht trivialer Werte etwa der Wurzel aus der Anzahl aller Werte ist.
Dies bedeutet im Umkehrschluss, dass im Schnitt jede Zeile einen neuen Faktor einführt. Die Summe nicht trivialer Werte weist bei allen Matrizen
mehr als 50$\%$ auf. 

\begingroup
  \renewcommand*{\arraystretch}{1.2} % Zeilenabstand der Tabelle
\begin{table}[ht!]
 \centering
 \caption{Bewertung der DCT-Twiddlefaktor-Matrizen}
 \begin{tabular}{lccccc}
   \hline  
   N                                                 & 8     & 9      & 12     & 15    & 16\\
   \hline
   N$\times$N                                        & 64    & 81     & 144    & 225   & 256\\
   \rowcolor{lightgray}
   $\sum$ trivialer Werte                            & 8     & 33     & 28     & 63    & 16\\
   \rowcolor{lightgray}
   $\sum$ nicht trivialer Werte                      & 56    & 48     & 116    & 162   & 240\\
   Anzahl verschiedener nicht trivialer Werte        & 7     & 7      & 10     & 13    & 15\\
   Verhältnis $\sum$ trivial / $\sum$ nicht trivial  & 0.143 & 0.6875 & 0.2414 & 0.389 & 0.067\\
   \hline
 \end{tabular}
 \label{tab:DCT-TwiddlefaktorMatrizenBewertung}  
\end{table}
\endgroup
