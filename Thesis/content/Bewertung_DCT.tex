\section{Bewertung verschiedener DFT- und DCT-Größen}\label{sec:BewertungVerschiedenerGroessen}
In diesem Abschnitt sollen Erkenntnisse gewonnen werden, auf denen basierend die Wahl der Transformation und die Größe ihrer Matrix getroffen werden kann.
Die Bewertung berücksichtigt wie bereits angedeutet die beiden Eigenschaften Anzahl verschiedener Faktoren und die gesamte Anzahl an Faktoren, wobei die erst genannte 
größeren Einfluss auf eine negative Bewertung hat, da sich (betragsmäßig) gleiche Faktoren mit Hilfe des Distributivgesetzes ausklammern lassen.
Bei den Faktoren wird zwischen solchen unterschieden, die als trivial erachtet werden, da sie nur einer Addition bedürfen ($\pm1$), zusätzlich
zur Addition nur eine Division durch 2 erfolgt ($\pm0,5$) oder gar keine Berechnung nötig ist ($0$) und solchen, die als nicht trivial betrachtet werden müssen, da eine Multiplikation
unumgänglich ist (beispielsweise $\tfrac{\sqrt{2}}{2}$). Begründet werden kann dies mit dem dualen Zahlensystem, da eine Multiplikation mit als trivial eingestuften Werten statt eines 
komplexen Schaltnetzes im aufwändigsten Fall Bitshifts erfordern.

Interessant erscheint die DFT, da die Sensoren je ein Kosinus- und ein Sinussignal ausgeben, welches sich zu einem komplexen Signal zusammenfassen lässt. Der Nachteil der komplexen Ausgangsmatrix realtiviert sich dadurch deutlich. Darüberhinaus sind bei der DFT Symmetrien sowohl innerhalb der Spalten als auch der Zeilen vorhanden sind. Die DCT weist Symmetrien nur innerhalb der Zeilen auf.


\subsection{Bewertung verschiedener DCT-Größen}
In Tabelle \ref{tab:DCT-TwiddlefaktorMatrizenBewertung} ist die Gegenüberstellung der genannten Größen zu sehen. Für die Bewertung wurde das 
Matlab-Skript aus Anhang \ref{src:dct_bewertung} geschrieben.
Ersichtlich ist, dass die Anzahl verschiedener nicht trivialer Werte etwa der Wurzel aus der Anzahl aller Werte ist.
Dies bedeutet im Umkehrschluss, dass im Schnitt jede Zeile einen neuen Faktor einführt. Die Summe nicht trivialer Werte weist bei allen Matrizen
mehr als 50$\%$ auf. 

\begingroup
  \renewcommand*{\arraystretch}{1.2} % Zeilenabstand der Tabelle
\begin{table}[ht!]
 \centering
 \caption{Bewertung der DCT-Twiddlefaktor-Matrizen}
 \begin{tabular}{lccccc}
   \hline  
   N                                                 & 8     & 9      & 12     & 15    & 16\\
   \hline
   N$\times$N                                        & 64    & 81     & 144    & 225   & 256\\
   \rowcolor{lightgray}
   $\sum$ trivialer Werte                            & 8     & 33     & 28     & 63    & 16\\
   \rowcolor{lightgray}
   $\sum$ nicht trivialer Werte                      & 56    & 48     & 116    & 162   & 240\\
   Anzahl verschiedener nicht trivialer Werte        & 7     & 7      & 10     & 13    & 15\\
   Verhältnis $\sum$ trivial / $\sum$ nicht trivial  & 0.143 & 0.6875 & 0.2414 & 0.389 & 0.067\\
   \hline
 \end{tabular}
 \label{tab:DCT-TwiddlefaktorMatrizenBewertung}  
\end{table}
\endgroup
