\section{Test des VHDL-Implementation der 2D-DFT}
Zuerst wurde die Matrixmultiplikation mit Faktoren entworfen und getestet, die sich leicht nachvollziehen lassen. Anschließend wurden die Faktoren 
mit den korrekten Werten ersetzt, sodass die berechneten Werte denen der 1D-DFT entsprechen. Zuletzt wurde das Programm um die Funktionalität 
einer transponiert wiederholten DFT ergänzt und getestet.

\subsection{Matrixmultiplikation mit ganzzahligen Faktoren, 1D-DFT und 2D-DFT}
Der Faktor $\frac{\sqrt{2}}{2}_{10}$ ist im Quelltext als 13\,Bit Vektor im S2Q10-Format abgelegt und hat den Wert $0001011010100_2$.
Das Waveform-Programm \texttt{SimVision} interpretiert diesen Wert als Ganzzahl. In der dezimalen Darstellung ist dies 724.
Um die Simulationsergbnisse nachvollziehen zu können, wurden im ersten Schritt die Werte  $\frac{\sqrt{2}}{2}$ der 
Twiddlefaktormatrix mit dem Wert \num{2.9296875}$\cdot10^{-3}$ ersetzt. Dieser Wert entspricht im S2Q10-Format $0000000000011$ und als
dezimale Ganzzahl 3. Hiermit lassen sich gut die in \texttt{SimVision} angezeigten Ergebnisse überprüfen.
Ein weiterer Vorteil ist, dass bei diesem Faktor und den gewählten Eingangswerten der Wertebereich der 12\,Bit Vektoren der Ausgangsmatrix ausreichen und es nicht zu Überläufen kommt.
Aus diesem Grund konnte zu diesem Zeitpunkt auf Bitshifts nach den Additionen und Multiplikationen verzichtet werden.
Erst als das binäre Äquivalent des Faktors $\frac{\sqrt{2}}{2}$ bzw. als Ganzzahl interpretiert 724 eingeführt wurde, hatten die Ergebnisse der Matrixmultiplikation mehr als die zwölf Stellen der Vektoren der Ausgangsmatrix.
Mit der Änderung der Faktoren wurde es deshalb erforderlich durch Bitshifts zu verhindern, dass die Werte größer als $2^{12}=4096$ werden.
Wann und wie viele Bitshifts erfolgen geht aus den beiden Abbildungen \ref{pic:AkkumulationUngeradeSpalten} und \ref{pic:AkkumulationGeradeSpalten} hervor.

Nachdem der Entwurf der 1D-DFT abgeschlossen war und die richtigen Ergebnisse berechnet wurden, wurde die Funktionalität der 2D-DFT ergänzt. 
Durch kleine Änderungen ist es leicht möglich den Programmablauf nach der Berechnung
der 1D-DFT zu beenden. 
Um zu testen, ob Änderungen am Quelltext, welche zu fehlerhafte Berechnungen der 2D-DFT führten, auch Einfluss auf die 1D-DFT haben, war dies sehr hilfreich.

Zum Test der IDFT muss das entsprechende Eingangssignal gesetzt sein. Nun werden die Zeilen der Twiddlefaktormatrix wie in Abschnitt \ref{sec:idft} beschrieben 
in umgekehrter Reihenfalge durchlaufen. Auf diese Weise lässt sich auf den mit der 2D-DFT errechneten Werten die Rücktransformation anwenden und abschließend mit den
ursprünglichen Werten vergleichen.

\subsection{Automatisierter Testdurchlauf}\label{sec:automatisierterTestdurchlauf}
Häufig ist nicht der Verlauf aller Signale von Interesse ist, sondern die auschließlich die der Ergebnismatrix. Diese lassen sich ebenfalls in \texttt{SimVision} betrachten.
Da dies nicht komfortabel ist, wurde von vornherein geplant die Ergebnisse zusätzlich die Datei \texttt{Results.txt} zu schreiben.
In dieser Datei werden die logischen Pegel \texttt{High} und \texttt{Low} durch eine 1 respektive eine 0 repräsentiert. Diese Daten lassen sich beispielsweise mit
dem Programm Matlab einlesen und mit der Funktion \texttt{fi()} in das gewünschte Dezimalformat konvertieren. Um die in Hardware erfolgten Bitshits zu kompensieren,
werden die Werte mit 256 multipliziert. Die so erhaltenen Werte lassen sich nun mit der auf die Eingangswerte angewandte 1D- bzw. 2D-DFT vergleichen. Dazu müssen
diese auf gleiche Weise eingelesen werden, wie die Ergbniswerte.
%Um in Matlab eine komplexe Matrix zu transponieren, muss entweder die Funktion \texttt{transpose()} oder \texttt{.'} verwendet werden. 

Damit nicht zwangsläufig der gesamte Test durchlaufen werden muss, sondern sich beispielsweise auf das Generieren neuer Ergbniswerte beschränkt werden kann,
erfolgt der Aufruf der VHDL Simulation und die Verifikation mit Matlab in getrennten Dateien. Die Namen der Skripte lauten \texttt{simulate.sh} und \texttt{VHDL\_DFT\_Vergleich.m}. 
Mit dem dritten Shell-Skript \texttt{check.sh} können beide aufeinanderfolgend ausgeführt werden. Der beschriebene Ablauf wird nachfolgend detaillierter und anschaulicher erläutert.

In Matlab kann die Twiddlefaktormatrix mit
%\begin{equation}\label{eq:matlab_dft_faktoren}
\[ W = e^{-\frac{i 2 \pi}{N}\cdot[0:N-1]'\cdot[0:N-1]} \]
%\end{equation}
berechnet werden, wobei N die Anzahl der Elemente je Zeile bzw. Spalte ist. 

\begin{mytextbox}{Bash - check.sh}
\vspace{0.5cm}
\# Dieses Skript ruft das Bash-Skript \texttt{simulate.sh} auf, mit dem neue Ergebnisdateien \\
\# erzeugt werden. Anschließend wird Matlab im Shell-Modus gestartet und das Skript \\
\# VHDL-DFT\_Vergleich.m ausgeführt.
\vspace{0.5cm}
 \begin{mytextbox}{Bash - simulate.sh}
 \vspace{0.5cm}
  \# Dieses Skript automatisiert die Simulation der 2D-DFT in VHDL.\\
  \# \texttt{ncsim} muss ein \texttt{.tcl}-Skript übergeben werden, in dem die Simulationsdauer \\ \# angegeben ist.
  
  \begin{itemize}
   \item kompilieren (alles)
   \item elaborieren (Top)
   \item simulieren (Top, alles) - \texttt{-f simulationTime.tcl} als Option übergeben
    \begin{itemize}
     \item Daten einlesen (\texttt{InputMatrix\_komplex.txt})
     \item Berechnen der 2D-DFT
     \item Ergebnisse in Datei schreiben (\texttt{Results.txt})
    \end{itemize}
    \vspace{0.5cm}
  \end{itemize}
 \end{mytextbox}
 \vspace{0.5cm}
 \begin{mytextbox}{Matlab - VHDL-DFT\_Vergleich.m}
\vspace{0.5cm}
  \% Dieses Skript vergleicht die Berechnung der 2D-DFT in VHDL mit der DFT 
  \% nach Gl. (\ref{eq:2D-DFT_MatrixMult}).
  
  \begin{itemize}
   \item Daten einlesen (\texttt{InputMatrix\_komplex.txt})
   \item Daten von S1Q10 nach Dezimal konvertieren
   \item Berechnen der 2D-DFT
   \item Daten einlesen (\texttt{Results.txt})
   \item Daten von S1Q10 nach Dezimal konvertieren
   \item Multiplikation mit 256 zur Kompensation des 8-fachen Bitshift
   \item Differenz beider Ergebnisse ausgeben
  \end{itemize}
 \end{mytextbox}
 \vspace{0.5cm}
\end{mytextbox}
