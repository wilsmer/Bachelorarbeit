\section{Diskrete Kosinus Transformation (DCT)}
\subsection{Verwendung}


\subsection{Berechnung}
Für die Berechnung der DCT gibt es verschiedene Varianten, welche sich in der Symmetrie der Ergebnismatrix unterscheiden. (Stimmt das wirklich? was sonst?)

Darüber hinaus wird in der Bildverarbeitung häufig die 1. Zeile der Twiddlefaktormatrix mit dem Faktor $\frac{1}{\sqrt2}$, sowie die gesamte Matrix mit 
$\sqrt{\frac{2}{N}}$, $N =$ Anzahl Elemente in einer Zeile bzw. Spalte, multipliziert.

Da es hier um eine Aufwandsabschätzung geht, wird sich auf die in der Bildverarbeitung gängigste Variante jedoch ohne die skalierenden Faktoren beschränkt.
Diese berechnet sich zu

\begin{equation}
X^*[k] = \sum_{n=0}^{N-1} x[n] \cos\left[\frac{\pi k}{N} \left(n+\frac{1}{2}\right) \right] \quad \textrm{für} \quad  k=0,\dots,N-1
\end{equation}

Die Twiddlefaktormatrix kann in Matlab mit
 %\lstinputlisting[language=matlab, caption={}, frame=no, numbers=none, label=src:dct_faktoren]{Skripte/Matlab/DCT_Faktoren.m}
 \begin{equation}\label{eq:matlab_dct_faktoren}
  W = \cos\left(\frac{\pi}{N}\cdot \left([0:N-1]')*([0:N-1]+\frac{1}{2}\right)\right)
 \end{equation}

berechnet werden.