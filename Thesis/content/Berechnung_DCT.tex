\section{Diskrete Kosinus Transformation (DCT)}
Die mit dem Dmeo-Array gewonnenen Daten lassen sich als Bild mit $8\times8=64$ Pixeln darstellen. Es ist denkbar anhand dieses Bildes die gesuchten
Informationen mittels Bildverarbeitungsalgorithmen zu errechnen.
Zu den häufigsten Algorithmen zählt hier die \gls{dct}, das sie aus einem reellen Eingangssignal eine ebenfalls reelle Transformierte erzeugt~\autocite[355-359]{burger2006digBildverarbeitung}. 
Ein weiterer Vorteil den sie gegenüber der \gls{dft} hat, ist dass bei der periodischen Wiederholung im Spektrum Unstetigkeiten aufgrund von Kanten umgangen werden können
~\autocite[28-31]{psturm1999bildkompression}.

Für die Berechnung der DCT gibt es verschiedene Varianten, die sich in der Symmetrie des Sepktrums unterscheiden. Die in der Bildverarbeitung häufigste ist 
die sogenannte \textit{ungerade DCT}~\autocite[28-31]{psturm1999bildkompression}.
Häufig wird die erste Zeile der Twiddlefaktormatrix mit dem Faktor $\nicefrac{1}{\sqrt2}$, sowie die gesamte Matrix mit 
$\sqrt{\frac{2}{N}}$, $N =$ Anzahl Elemente in einer Zeile bzw. Spalte, skaliert~\autocite{wikiDCT}. Auf diese Weise ergibt sich eine ortohogonale Matrix, bei der Inverse und
 Transponierte identisch sind.  

Diese Variante der DCT berechnet wie folgt:

\begin{equation}\label{eq:dct}
X^*[k] = \sum_{n=0}^{N-1} x[n] \cos\left[\frac{\pi k}{N} \left(n+\frac{1}{2}\right) \right] \quad \textrm{für} \quad  k=0,\dots,N-1
\end{equation}

Wie Gleichung (\ref{eq:dct}) entnommen werden kann, basiert die Transformation nur auf Kosinusfunktionen, weshalb auch das Spektrum rein reell ist.
% Die Twiddlefaktormatrix kann in Matlab mit
%  %\lstinputlisting[language=matlab, caption={}, frame=no, numbers=none, label=src:dct_faktoren]{Skripte/Matlab/DCT_Faktoren.m}
%  \begin{equation}\label{eq:matlab_dct_faktoren}
%   W = \cos\left(\frac{\pi}{N}\cdot \left([0:N-1]')*([0:N-1]+\frac{1}{2}\right)\right)
%  \end{equation}
% 
% berechnet werden.