\section{Diskrete Fouriertransformation (DFT)}




\subsection{Verwendung}

Die \gls{dft} (Gl. \ref{eq:dft}) ist die zeit- und wertdiskrete Variante der \gls{ft}, die statt von $-\infty$ bis $\infty$ nur von 0 bis N-1 Elemente läuft. 
Im Frequenzspektrum wiederholt sich 
Da es sich um diskrete Werte handelt, geht das Integral in eine endliche Summe über. Die 2D-DFT ist nur möglich (sinnvoll), wenn die Eingangswerte in Form eines Vektors vorliegen.


\subsection{Summen- und Matrizenschreibweise der DFT}
\subsubsection{1D-DFT}
Die \gls{1d-dft} findet wie bereits erwähnt üblicherweise Anwendung, um vom Zeit- in den Frequenzbereich zu gelangen.
\begin{equation}\label{eq:dft}
 X^* \left[ m \right] = \frac{1}{N} \cdot \sum^{N-1}_{n=0} x[n] \cdot e^{-\frac{j 2 \pi m n}{N}}
\end{equation}



Gleichung \ref{eq:1D-DFT_MatrixMult} zeigt die obige Summenformel umgeschrieben zu einer Matrixmultiplikation.

Mit Gleichung \ref{eq:Twiddlefaktorenberechnung} werden zunächst alle Twiddlefaktoren in Matrixform berechnetet, wobei n der Index des zu Berechnenden Elements des Vektors im Zeitbereich und
m das Äquivalent im Frequenzbereich ist.
\begin{equation}\label{eq:Twiddlefaktorenberechnung}
\sum^{N-1 }_{m=0} \sum^{N-1 }_{n=0} e^{-\frac{j 2 \pi m n}{N}} = W
\end{equation}

Somit gilt:

\begin{equation}\label{eq:1D-DFT_MatrixMult}
X^* = W \cdot x
\end{equation}


\subsubsection{2D-DFT}
Die \gls{2d-dft} wird hingegen häufig in der Bildverarbeitung verwendet, um vom Orts- in den Fourierraum zu gelagen. Da es sich somit nicht mehr um eine Abhänigkeit 
der Zeit handelt, werden andere Indizes verwendet.
\begin{align}
\begin{split}
X[u,v] 	&= \frac{1}{N} \sum^{N-1}_{n=0} X^* \left[ m \right] \cdot e^{-\frac{j 2 \pi m n}{N}}\\
	&= \frac{1}{MN} \sum^{M-1}_{m=0} \left( \sum^{N-1}_{n=0} f(m,n) \cdot e^{-\frac{j 2 \pi m n}{N}} \right) \cdot e^{-\frac{j 2 \pi m n}{M}}
\end{split}
\end{align}

Auch hier lässt sich die Berechnung in Matrizenschreibweise darstellen:

\begin{align}
\begin{split}
 X &= W \cdot x\left(t\right) \cdot W \\
                    &= X^* \cdot W
\end{split}
\end{align}



\subsection{Inverse DFT}

Die \gls{idft} wird analog zur \gls{dft} mit 

\begin{equation}\label{eq:idft}
 x \left[ n \right] = \frac{1}{N} \sum^{N-1}_{n=0} X[m] \cdot e^{\frac{j 2 \pi m n}{N}}
\end{equation}

beschrieben. Durch die umgekehrte Drehrichtung des komplexen Zeigers werden in der Matrizenschreibweise die Zeilen 1 und 7, 2 und 6 sowie 3 und 5 vertauscht.



