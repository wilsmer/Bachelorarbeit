\subsection{Berechnung der Diskreten Fouriertransformation mittels FFT}\label{sec:BerechnungFFT}
Die Mathematiker Cooley und Tukey haben einen Algorithmus entwickelt und im Jahr 1965 veröffentlich, mit dem sich die \gls{dft} mit vergleichsweise wenig Multiplikationen
und somit deutlich schneller als bei der allgemeinen \gls{dft} berechnen lässt. Das Verfahren wird als \gls{fft} bezeichnet.
Grundlage ist, dass sich eine DFT
in kleinere Teil-DFTs aufspalten lässt, welche durch Ausnutzen von Symmetrieeigenschaften in der Summe weniger Koeffizienten haben. 
Üblich ist die Radix-2 FFT, Ausgangspunkt ist also eine DFT mit 2 Eingangswerten.
Da mit jeder weiteren Teil-DFT sich die Anzahl der Eingangswerte verdoppelt, eignet sich diese Methode nur für Eingangsvektoren der Größe $2^n$. Dieser
vermeindliche Nachteil lässt sich durch Auffüllen des Eingangsvektors mit Nullen (Zeropadding) eliminieren. Dies hat zur Folge, dass die Größe des Ausgangsvektors
immer eine Potenz von zwei ist. Abbildung \ref{pic:Butterfly} illustriert dies anhand eines Eingangsvektors mit acht Werten. 
Um diesen Algorithmus anwenden zu können ist es erforderlich, dass die Werte im Eingangsvektor in umgekehrte Bitreihenfolge getauscht werden (bitreversed order).
Dies geschieht nach dem Muster, dass die Indizes der Eingangswerte, wie
üblich bei 0 beginnend, binär dargestellt werden. Nun wird die Reihenfolge der Bits getauscht. Auf diese Weise tauschen bei einem 8-Bit Vektor die
Elemente 2 und 5 sowie 4 und 7 ihre Position. Andernfalls sind die Ergebnisse in vertauschter Reihenfolge.

Die Anzahl der benötigten komplexen Multiplikationen $m_{FFT}$ kann mit der Gleichung (\ref{eq:FFT_komplexMult}) abgeschätz werden.


\begin{equation}\label{eq:FFT_komplexMult}
 m_{FFT} = \frac{N}{2}\log_2(N)
\end{equation}







\begin{figure}[htbp]
 \centering
 \includegraphics[width=0.7\textwidth]{img/Butterfly.png}
 \caption{Berechnungsschema der DFT mit 8 Eingangswerten nach dem Butterfly-Verfahren}
 \label{pic:Butterfly}
\end{figure}


