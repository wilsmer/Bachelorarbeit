\subsection{Berechnung der schnellen Fouriertransformation}\label{sec:BerechnungFFT}
Die Mathematiker Cooley und Tukey haben einen Algorithmus entwickelt und im Jahr 1965 veröffentlicht, mit dem sich die \gls{dft} mit weniger Multiplikationen
und dadurch schneller als bei der allgemeinen \gls{dft} berechnen lässt. Das Verfahren wird als \gls{fft} bezeichnet.
Grundlage ist, dass sich eine DFT
in kleinere Teil-DFTs aufspalten lässt, welche durch Ausnutzen von Symmetrieeigenschaften in der Summe weniger Koeffizienten haben. 
Üblich ist die Radix-2 FFT, Ausgangspunkt ist eine DFT mit 2 Eingangswerten. 
Diese Methode kann nur auf Eingangsvektoren der Größe $2^n$ angewandt werden. Dieser
vermeintliche Nachteil lässt sich durch Auffüllen des Eingangsvektors mit Nullen (Zeropadding) eliminieren. Dies hat zur Folge, dass die Größe des Ausgangsvektors
immer eine Potenz von zwei ist. 
Die Anzahl der benötigten komplexen Multiplikationen $m_{FFT}$ kann mit der Gleichung (\ref{eq:FFT_komplexMult}) abgeschätzt werden.


\begin{equation}\label{eq:FFT_komplexMult}
 m_{FFT} = \frac{N}{2}\log_2(N)
\end{equation}



Das Verfahren wird in Kürze in Kapitel \ref{sec:AnalyseFFT} behandelt und kann unter anderem in dem Buch 
\textit{Digital Signal Processing: Principles, Algorithms and Applications}~\autocite{john2007digital} auf den Seiten 511 bis 524 nachgelesen werden.