 \section{Entscheidung über Größe und Art der Transformation}
 Noch nicht fertig!
 
 Sowohl die \gls{dct} als auch die \gls{dft} finden häufig in der Bildverarbeitung Anwendung. Der Vorteil der \gls{dct} gegenüber der \gls{dft} ist,
 dass sie rein reelle Ergebniswerte liefert. Ihr großer Nachteil zeigt sich u.a. insbesondere deutlich bei den 8x8-Matrizen, da sich hier 
 
 
 nicht trivial darstellbare Zahlen der DCT einem einzigen bei der 8x8-DFT gegenüber stehen.
 

Auch wenn bei der DFT mit der Berechnung des imaginären Teils zusätzlicher Implementierungsaufwamd hinzukommt, wird davon ausgegangen, dass dieser geringer ist, 
 als alle x Multiplikationen umzusetzen. Ebenso ist die Annahme, dass der Platzbedarf auf einem Chip in einer ähnlichen Größenordnung liegt, da auf der einen Seite
 der zusätzliche Speicherbedarf für eine weitere Matrix den x Konstantenmultiplizierer-Schaltnetzen gegenüber stehen.
 
 Es ist nicht geklärt, welche Berechnung für eine Weiterverarbeitung sinnvoller ist. Dies heraus zu finden ist jedoch nicht Bestandteil der Aufgabenstellung dieser Arbeit.
 An dieser Stelle sollen lediglich Vor- und Nachteile zusammengetragen werden, die eine Entscheidung rechtfertigen.
 
 Ein Einsatzszenario der Transformationen ist die Filterung von Rauschen und anderen Störgrößen. Hierfür ist die DFT gut geeignet. 
 
 
 Da es bei dieser Arbeit vor allem um die Aufwandsabschätzung einer optimierten Matrizenmultiplikation zur Vorverarbeitung der Sensordaten geht, 
 welche als Ausgangspunkt für eine finale Implementation dient, und es sich hier um keine endgültige Entscheidung handelt, ist die DFT gut geeignet.
 
 
 \begin{table}[ht]
 \centering
  \caption{Gegenüberstellung der Vor- und Nachteile von DCT und DFT}
  \begin{tabular}{lcc}
  \hline
      Eigenschaft          & Vorteil   & Nachteil\\
  \hline
   Imaginärteil Vorhanden  & DCT       & DFT \\
   Anzahl Multiplikationen & DFT       & DCT\\
   Platzbedarf             &  -        &  - \\
   \hline
  \end{tabular}
  \label{tab:gegenüberstellung_dct_dft}
 \end{table}
