\section{Simulation}
 \subsection{NC Sim - positive Zahlendarstellung}
 
 \section{Anzahl benötigter Takte}
 Anhand der Simulation kann die Anzahl der vorausgesagten benötigten Takte verifiziert werden. 
 
 Nachdem \texttt{nReset} auf '1' gesetzt wird, werden die Eingangswerte
 eingelesen. Wenn dieser Vorgang abgeschlossen ist, geht \texttt{loaded} auf '1'. Mit der nächsten steigenden Taktflanke, in Bild \ref{pic:Simulationsdauer} bei 
 \SI{340}{ns}, beginnt die Berechnung
 der \gls{2d-dft}. Beendet ist sie, nachdem die Matrizenmultiplikation auf die Eingangswerte und anschließend auf die \gls{1d-dft}-Werte angewandt wurde. Also nach $2 \cdot 64$
 einzelnen Berechnungen. Wenn dies erfolgt ist, wird \texttt{result\_ready} auf '1' gesetzt. Dies geschieht bei \SI{20\,820}{ns}. Bei einer Taktfrequenz von $(\SI{40}{ns})^{-1}$
 (siehe \ref{src:dft8_optimiert_top}) ergeben sich so 512 Takte. Dies bestätigt auch der Edge Count, ebenfalls auf dem Bild zu sehen, welcher die Flanken des \texttt{clk}-Signals 
 zählt. In der Simulation ist zu erkennen, dass die Berechnung der Elemente 
 unterschiedlich viele Takte beansprucht. Hieran lässt sich ebenfalls sehen, dass die 1. (ungerade) Zeile weniger Takte gegenüber der 2. (geraden) Zeile benötigt. 
 
 %Auch in der Abbildung \ref{pic:Simulationsdauer} zu sehen ist, dass \texttt{element\_out} für 0 bis 7 weniger Takte einnimmt, als in den darauf folgenden 8. Dieses Muster
 %wiederholt sich und hat, wie in Abschnitt \ref{sec:berechnung_anzahl_takte} erläutert, damit zu tun, dass für die geraden
 
 \begin{figure}[htbp]
  \centering
  \includegraphics[width=0.58\textwidth]{img/Simulationsdauer_Anfang.png}
  \hfill
  \includegraphics[width=0.161\textwidth]{img/Simulationsdauer_Mitte.png}
  \hfill
  \includegraphics[width=0.241\textwidth]{img/Simulationsdauer_Ende.png}
  \caption{Simulations der 2D-DFT mit \texttt{NC Launch}}
  \label{pic:Simulationsdauer}
 \end{figure}

 \begin{figure}[htbp]
  \centering
  \includegraphics[width=0.6\textwidth]{img/Simulation_edge_count_clk.png}
  \caption{Edge Count für eine 2D-DFT}
 \end{figure}

 \section{Zeitabschätzung im Einsatz als ABS-Sensor}
 Anhand der nun bekannten Größe von 512 Takten kann ermittelt werden, ob diese Implemenatation vom zeitlichen Aspekt her akzeptabel ist.
 Da ein Einsatzszenario der ABS-Sensor ist, wird an dieser Stelle ein Blick hierauf geworfen. Da der ABS-Sensor an der Radnabe sitzt, wird 
 hierfür die Raddrehzahl benötigt. Um diese zu ermitteln, wird von einer maximalen Geschwindikeit von $v_{max}$ = 250\,KM/h ausgegangen. 
 Weiter wird ein realtiv kleiner Reifenumfang von ca. 1\,m angenommen. Als maximale Taktfrequenz des Sensors ist 1\,MHz vorgegeben.
 
 Der Reifen hat eine Breite von 175 cm, eine Flankenhöhe von 75\,$\%$ der Breite und die Felge einen Durchmesser von 14 Zoll. Somit errechnet sich der Reifenumfang
 gemäß (\ref{eq:Reifenumfang})
 
 \begin{align}\label{eq:Reifenumfang}
 \begin{split}
  U &= (\SI{175}{cm} \cdot 75\% \cdot 2 + 14 \cdot \SI{2.54}{cm})\cdot \pi\\
    &\simeq \SI{0,94}{m}
 \end{split}
 \end{align}

 In Gleichung \ref{eq:Umdrehungen} wird die Anzahl der Radumdrehungen bei maximaler Geschwindigkeit berechnet
 
 \begin{equation}\label{eq:Umdrehungen}
  \begin{split}
   RPM &= \dfrac{\dfrac{\SI{250}{Km/h}}{\SI{0,94}{m}}}{\SI{60}{sec}}\\
       &= \SI{4386}{\frac{U}{min}}\\
       &= \SI{73}{\frac{U}{sec}}
  \end{split}
 \end{equation}

 Durch die Taktfrequenz und die benötigten Takte kann in (\ref{eq:dft_sekunde}) die maximale Anzahl der 2D-DFTs pro Sekunde errechnet werden.
 
 \begin{equation}\label{eq:dft_sekunde}
  \begin{split}
   N_{DFT, sec} &= \frac{\SI{100}{MHz}}{\SI{512}{Takte}}\\
                &= 195312
  \end{split}
 \end{equation}

 Somit ist es nun möglich die unter diesen Voraussetzungen maximale Zahl der 2D-DFTs während einer Umdrehung zu bestimmen (\ref{eq:max_dft_umdrehung})
 
 \begin{equation}\label{eq:max_dft_umdrehung}
  \begin{split}
   N_{DFT,U}  &= \frac{\SI{195312}{\frac{2D-DFT}{sec}}}{\SI{73}{\frac{U}{sec}}}\\
              &= \SI{2675}{\frac{2D-DFT}{U}}
  \end{split} 
 \end{equation}

 Nun kann in (\ref{eq:max_dft_winkel}) gezeigt werden, dass bei einer Winkelauflösung von $1^\circ$ knapp 7,5 2D-DFTs berechnet werden könnten. Die Dauer liegt somit 
 gut im zeitlichen Rahmen, der vorganden ist. Darüber hinaus kann an dieser Stelle bereits gesagt werden, dass noch reichlich Zeit für andere Berechnungen vorhanden ist.
 
 \begin{equation}\label{eq:max_dft_winkel}
  \begin{split}
   N_{DFT,1^\circ} &= \frac{\SI{2675}{\dfrac{2D-DFT}{U}}}{360^\circ}\\
                   &= \SI{7,43}{\dfrac{2D-DFT}{1^\circ}}
  \end{split}
 \end{equation}

 
 Um eine Aussage über die restliche zur Verfügung stehenden Zeit bzw. Takte machen zu können, wird in Gleichung (\ref{eq:takte_pro_winkel}) gezeigt, dass pro Winkel 
 etwa 3800 Takte für Berechnungen zu Verfügung stehen. Somit ist gezeigt, dass für andere Aufgaben ausreichen Zeit vorhanden ist und die Implemenatation 
 erfolgreich ist.
 
 \begin{equation}\label{eq:takte_pro_winkel}
  \begin{split}
   N_{Takte, U} &= \frac{\SI{100}{MHz}}{73\dfrac{U}{sec}}\\
                &= 1,37\cdot 10^6 \frac{Takte}{Umdrehung}\\
   N_{Takte, 1^\circ} &= \frac{1,37\cdot 10^6 \dfrac{Takte}{Umdrehung}}{360^\circ}\\
                      &\simeq 3800 \ \textrm{Takte}
  \end{split}
 \end{equation}

 Da 512 etwa 13,5$\%$ von 3800 sind, resultiert hieraus, dass noch etwa 86,5$\%$ bzw. knapp 3300 Takte nutzbar sind.

 
 