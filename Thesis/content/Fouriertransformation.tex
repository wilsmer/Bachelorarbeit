\section{Fourierreihenentwicklung}
Mit einer Fourierreihe kann ein periodisches, abschnittsweise stetiges Signal aus einer Summe von Sinus- und Konsinusfunktionen
zusammengesetzt werden. Die Schreibweise als Summe von Sinus- und Kosinusfunktionen (Gl. \ref{eq:Fourierreihenentwicklung}) ist eine 
der häufigsten Darstellungsformen.

\begin{equation}\label{eq:Fourierreihenentwicklung}
 x(t) = \frac{a_0}{2} + \sum_{k=1}^\infty \left(a_k cos(kt) + b_k sin(kt)\right)
\end{equation}

Die Fourierkoeffizienten lassen sich über die Gleichungen (\ref{eq:a_k}) und (\ref{eq:b_k}) berechnen:

\begin{equation}\label{eq:a_k}
 a_k = \frac{1}{\pi} \int_{-\pi}^{\pi} x(t) \cdot cos(kt) dt \quad \textrm{für} \quad k \geq 0 
\end{equation}
\begin{equation}\label{eq:b_k}
 b_k = \frac{1}{\pi} \int_{-\pi}^{\pi} x(t) \cdot sin(kt) dt \quad \textrm{für} \quad k \geq 1
\end{equation}

Mit der Exponentialschreibweise lassen sich Sinus und Kosinus auch wie in (\ref{eq:cos_exp}) und (\ref{eq:sin_exp}) ausdrücken:

\begin{equation}\label{eq:cos_exp}
 cos(k t) = \frac{1}{2}\left(e^{j k t} + e^{-j k t} \right)
\end{equation}

\begin{equation}\label{eq:sin_exp}
 sin(k t) = \frac{1}{2j}\left(e^{j k t} - e^{-j k t} \right)
\end{equation}

und zusammengefasst ergibt sich in (Gl. \ref{eq:komplexerZeiger}) der komplexe Zeiger, der eine Rotation im Gegenuhrzeigersinn auf dem Einheitskreis beschreibt.
In Abbildung \ref{pic:Einheitskreis} dies zusätzlich noch grafisch dargestellt.

 \begin{align}
\begin{split}\label{eq:komplexerZeiger}
cos(k t) +j\cdot sin(k t) &= \frac{1}{2}\left(e^{j k t} + e^{-j k t} \right)+j\cdot \frac{1}{2j}\left(e^{j k t} - e^{-j k t} \right)\\
&= \frac{1}{2} \left(e^{j k t} + e^{j k t}\right)\\
&= e^{j k t}\\
\end{split}
\end{align}


\tikzstyle{dot}=[draw,shape=circle]
\begin{figure}[ht]
\centering
\begin{tikzpicture}[scale=2.5]
\draw[step=.5cm,gray, thin];
\draw (-1.2,0) -> (1.2,0);
\draw (0,-1.2) ->(0,1.2);
\draw (0,0) circle[radius=1cm];
\draw [thick, dashed] (0,0.5) -- (0.866, 0.5);
\draw [thick, dashed] (0.866, 0) -- (0.866, 0.5);
\draw [thick] (0, 0) -- (0.866, 0.5);
\draw (1.1, 0.1) node {1};
\draw (1, 0.6) node {$e^{jkt}$};
\draw (1.1, 0.1) node {1};
\draw (-0.07, 1.1) node {$j$};
\draw (-1.1, 0.1) node {-1};
\draw (-0.15, -1.1) node {$-j$};
\node[dot, fill, inner sep=1pt] at(0.866,0.5){};
\draw [decorate,decoration={gray,brace,amplitude=5pt,raise=2pt},yshift=0pt](0,0) -- (0,0.5) node [rotate=90,black,midway,yshift=0.6cm]{$sin (kt)$};
\draw [decorate,decoration={brace,amplitude=5pt,mirror,raise=2pt},yshift=0pt](0,0) -- (0.866,0) node [black,midway,yshift=-0.6cm]{$cos (kt)$};
\end{tikzpicture}
\caption{Einheitskreis, Zusammensetzung des komplexen Zeigers aus Sinus und Kosinus}
\label{pic:Einheitskreis}
\end{figure}


Die Fourierkoeffizienten $a_k$ und $b_k$ lassen sich auch als komplexe Zahl $c_k$ zusammengefasst berechnen:

\begin{equation}
 c_k = \frac{1}{2\pi} \int_{-\pi}^{\pi} x(t) e^{-j2\pi kt} dt \quad \forall k \in \mathbb{Z}
\end{equation}



\begin{equation}
 x(t) = \sum_{-\infty}^{\infty} c_k e^{jkt}
\end{equation}


\section{Fouriertransformation}
Mit der Fouriertransformation kann ein periodisches, abschnittsweise stetiges Signal $f(x)$ in eine Summe aus Sinus- und
Kosinusfunktionen unterschiedlicher Frequenzen zerlegt werden. Da diese Funktionen jeweils mit nur einer Frequenz periodisch sind, entsprechen diese
Frequenzen den Frequenzbestandteilen von $f(x)$. 

Grundlage für die Fouriertransformation ist das Fourierintegral (Gl. \ref{eq:Fouriertransformation})

\begin{equation}\label{eq:Fouriertransformation}
 X(f) = \int^{\infty}_{-\infty} x(t) \cdot e^{-j 2 \pi f t}
\end{equation}

Wenn Sinus und Kosinus wie in Gl. (\ref{eq:cos_exp}) und (\ref{eq:sin_exp}) als Exponentialfunktion geschrieben werden,
können sie zu einer komplexen Exponentialfunktion zusammengefasst werden.






Für komplexere Signale, etwa ein Rechteck, ergeben sich entsprechend sehr viele dieser Frequenzbeiträge. Deren Höhe ist Information darüber, wie groß ihr Anteil, also die Amplitude des 
Zeitsignals, ist. Die Fouriertransformation kann als das Gegenteil der Fourierreihenentwicklung gesehen werden.


- unendliche Dauer -> Leistungssignal?

- endliche Dauer -> Energiesignal?



Energiesignal:

Leistungssignal: Signal unendlicher Energie, aber mit endlicher mittlerer Leistung

Ein Zeitsignal hat ein eindeutig zuordbares Frequenzsignal (bijektiv), abgehsehen von Amplitude? und Phase

Spektrum: Frequenzbestandteile eines Signals

Berechnung des Spektrums: Spektralanalyse, Frequenzanalyse


In der Praxis, also basierend auf echten Messdaten, wird die die Bestimmung des Spektrums Spektrumschätzung genannt.


In der vorliegenden Arbeit wird künftig $X^*$ für die 1D-DFT und $X$ für die 2D-DFT stehen.

